\documentclass[krantz1,ChapterTOCs]{krantz}
\usepackage{fixltx2e,fix-cm}
\usepackage{amssymb}
\usepackage{amsmath}
\usepackage{graphicx}
\usepackage{subfigure}
\usepackage{makeidx}
\usepackage{multicol}
\usepackage{hyperref}
\usepackage{xcolor}

\begin{document}

\section{Exercises-Answer key}

\begin{enumerate}
    \item Suppose $\samplespace = \{ a,b,c \}$ and $P(\{a\})=0.2$, $P(\{b\})=0.3$, $P(\{c\})=0.5$. Define a random variable $X$ such that $X(a)=1$, $X(b)=1$, and $X(c)=0$.
    Define a second random variable $Y$ such that $Y(a)=0$, $Y(b) = 1$, $Y(c)=2$.
    \begin{enumerate}
        \item Compute $P(X=1)$
        \begin{enumerate}
            \item {\color{red} P(X=1) = 0.2+0.3=0.5   }
        \end{enumerate}
        
        \item Compute $P(X=0)$
        \begin{enumerate}
            \item {\color{red} P(X=0) = 0.5   }
        \end{enumerate}
        
        \item What is $supp(X)$ ? 
        \begin{enumerate}
            \item {\color{red} supp(X) = \{0,1\}   }
        \end{enumerate}
        
        \item What new sample space does $X$ generate?
        \begin{enumerate}
            \item {\color{red} \mathcal{G} = supp(X) = \{0,1\}   }
        \end{enumerate}

        \item Compute $P(Y=1)$
        \begin{enumerate}
            \item {\color{red} P(Y=1) = 0.3   }
        \end{enumerate}

        \item Compute $P(Y=0)$
        \begin{enumerate}
            \item {\color{red} P(Y=0) = 0.2   }
        \end{enumerate}
        
        \item What is $supp(Y)$ ? 
        \begin{enumerate}
            \item {\color{red} supp(Y) = \{0,1,2\}   }
        \end{enumerate}

        \item What new sample space does $Y$ generate?
        \begin{enumerate}
            \item {\color{red} \mathcal{G} = supp(Y) = \{0,1,2\}   }
        \end{enumerate}

    \end{enumerate}
    \item Let $\samplespace = \{1,2,3,4,5,6,7,8,9,10,11\}$, the set of all positive integers. Further define a random variable $K$ with the following probability mass function 
    \begin{align*}
        f(k) = \left( \frac{1}{2} \right) ^{k} & \text{when } k \leq 10\\
        f(11) = 0.009
    \end{align*} 
    \begin{enumerate}
        \item Is the pmf $f$ a valid probability distribution? Why or why not?
        
        \begin{enumerate}
            \item {\color{red} Yes, the pmf is a valid probability distribution because the probability of one of the outcomes from 1 to 11 is equal to 1 and because no assignments by the pmf are negative.   }
        \end{enumerate}

        \item What value of $K$ is assigned the highest probability? 
        \begin{enumerate}
            \item {\color{red}  K=1. P(K=1) = 1/2  }
        \end{enumerate}
        
        \item Please define the cumulative mass function for the random variable $K$
        
        \begin{enumerate}
            \item {\color{red} 
                \begin{align}
                    F(k) = \begin{cases}
                            1/2 & if k = 1 \\
                            1/2 + (1/2)^2 = 3/4    & if k = 2 \\
                            1/2 + (1/2)^2 + (1/2)^{3} = 7/8    & if k = 3 \\
                            0.9375    & \text{ if } k = 4 \\
                            0.9688    & \text{ if } k = 5 \\
                            0.984    & \text{ if } k = 6 \\
                            0.992    & \text{ if } k = 7 \\
                            0.996    & \text{ if } k = 8 \\
                            0.998    & \text{ if } k = 9 \\
                             0.999   & \text{ if } k = 10 \\
                            1        & \text{ if } k = 11 \\
                           \end{cases}
                \end{align} 
            
            }
        \end{enumerate}

        
    \end{enumerate}
    
    \item Define a random variable with $supp(Y) = \{-3,-2,-1,0,1,2,3\}$ and cumulative mass function 
    \begin{align*}
        F(y) = \begin{cases}
                  0.10  & \text{ when } y = -3\\
                  0.24  & \text{ when } y = -2\\
                  0.36  & \text{ when } y = -1\\
                  0.50  & \text{ when } y =  0\\
                  0.67  & \text{ when } y =  1\\
                  0.78  & \text{ when } y =  2\\
                  1.00  & \text{ when } y =  3\\
               \end{cases}
    \end{align*} 
    \begin{enumerate}
        \item What is $P(Y \leq -1)$
        \begin{enumerate}
            \item {\color{red} 0.10+0.24+0.36 = 0.7   }
        \end{enumerate}
        
        \item What is $P(Y=-1)$
        \begin{enumerate}
            \item {\color{red} 0.36-0.24=0.12   }
        \end{enumerate}
        \item What is $P(Y= 1)$
        \begin{enumerate}
            \item {\color{red} 0.67-0.50=0.17   }
        \end{enumerate}
        \item Please define the p.m.f for the random variable $Y$.
        \begin{enumerate}
            \item {\color{red} 
            \begin{align*}
                 f_{Y}(y) = \begin{cases}
                            0.10  & \text{ if } y=-3\\
                            0.14  & \text{ if } y=-2\\
                            0.12  & \text{ if } y=-1\\
                            0.14  & \text{ if } y=0\\
                            0.17  & \text{ if } y=1\\
                            0.12  & \text{ if } y=2\\
                            0.22  & \text{ if } y=3\\
                            \end{cases}
            \end{align*}
            }
        \end{enumerate}
        \item Graph the c.m.f
        \begin{enumerate}
            \item {\color{red} Graph of the above   }
        \end{enumerate}
        \item Graph the p.m.f
        \begin{enumerate}
            \item {\color{red} Graph of the above   }
        \end{enumerate}
    \end{enumerate}

    \item  Define the following joint distribution of random variable $Q$ and $R$ that are mapped from the sample space $\mathcal{G}$
    \begin{table}[ht!]
    \centering
    \begin{tabular}{c c | c}
        Q & R & prob \\
        \hline
        2 & 1 & 0.01\\
        2 & 2 & 0.075\\
        2 & 3 & 0.13\\
        1 & 1 & 0.1\\
        1 & 2 & 0.05\\
        1 & 3 & 0.17\\
        0 & 1 & 0.05\\
        0 & 2 & 0.30\\
        0 & 3 & 0.115
    \end{tabular}
    \end{table}

    \begin{enumerate}
        \item What is the implied support of $Q$?
        \begin{enumerate}
            \item {\color{red} $supp(Q) = \{0,1,2\}$   }
        \end{enumerate}
        
        \item What is the implied support of $R$?
        \begin{enumerate}
            \item {\color{red} $supp(R) = \{1,2,3\}$   }
        \end{enumerate}
        
        \item Compute P(Q=1,R=2)
        \begin{enumerate}
            \item {\color{red} $P(1,2) = 0.05$   }
        \end{enumerate}
        
        \item Compute the marginal probabilities for $Q$
        \begin{enumerate}
            \item {\color{red} 
            \begin{align*}
                P(Q=0) &= 0.465\\
                P(Q=1) &= 0.320\\
                P(Q=2) &= 0.215\\
            \end{align*}
            }
        \end{enumerate}
        
        \item Compute the marginal probabilities for $R$
        \begin{enumerate}
            \item {\color{red} 
            \begin{align*}
                P(R=1) &= 0.16\\
                P(R=2) &= 0.425\\
                P(R=3) &= 0.415\\
            \end{align*}
            }
        \end{enumerate}
        
        \item Compute $P(R=1 | Q=0)$
        \begin{enumerate}
            \item {\color{red} 
            \begin{align*}
                P(R=1,Q=0) / P(Q=0) = 0.05/0.465 = 0.107
            \end{align*}
            }
        \end{enumerate}

        \item The random variable $Q$ is called \textbf{statistically independent} from $R$ if for every value $q \in supp(Q)$ and for every value $r \in R$ the following is true $P(Q=q |R=r) = P(Q)$. Is $Q$ statistically independent from $R$? Why or why not?
        \begin{enumerate}
            \item { \color{red} 
            \begin{align*}
                R \text{ is not statistically independent from Q}\\       P(R|Q) \neq P(R)
            \end{align*}
            }
        \end{enumerate}
        
    \end{enumerate}
    \item For two events $A$ and $B$ that are statistically independent, we found that $P(A \cap B) = P(A)P(B)$. Please derive an equivalent expression for two random variables by applying the definition of conditional probability and statistical independence to random variables.

    \begin{enumerate}
            \item {\color{red} 
            \begin{align*}
                P(R|Q) = P(Q,R)/P(Q)\\
                P(Q)P(R|Q) = P(Q)P(R)\\
            \end{align*}
            }
    \end{enumerate}


    \item Define the following joint distribution of random variable $Q$ and $R$ that are mapped from the sample space $\mathcal{G}$
    \begin{table}[ht!]
    \centering
    \begin{tabular}{c c | c}
        Q & R & prob \\
        \hline
        2 & 1 & 0.01\\
        2 & 2 & 0.075\\
        2 & 3 & 0.13\\
        1 & 1 & 0.1\\
        1 & 2 & 0.05\\
        1 & 3 & 0.17\\
        0 & 1 & 0.05\\
        0 & 2 & 0.30\\
        0 & 3 & 0.115
    \end{tabular}
    \end{table}
     \begin{enumerate}
         \item Please compute $\mathbb{E}(Q)$
         
         \begin{enumerate}
             \item {\color{red}  P(Q) = \{(0,0.465),(1,0.32),(2,0.215)\}}
             \item {\color{red}  E(Q) = 0*0.465+1*0.32+2*0.215 = 0.75}
         \end{enumerate}
         
         \item Please compute $\mathbb{E}(R)$
         \begin{enumerate}
             \item {\color{red}  P(R) = \{(1,0.16),(2,0.425),(3,0.415)\}}
             \item {\color{red}  E(R) = 1*0.16+2*0.425+3*0.415 = 2.255}
         \end{enumerate}

         \item Please compute $V(Q)$
         \begin{enumerate}
             \item {\color{red}  V(Q) = 0.465*(0-0.75)^2 + 0.32*(1-0.75)^2 + 0.215*(2-0.75)^2 = 0.618    }
         \end{enumerate}
         
         \item Please compute $V(R)$
         \begin{enumerate}
             \item {\color{red}  V(R) = 0.160*(1-2.255)^2 + 0.425*(2-2.255)^2 + 0.415*(3-2.255)^2 = 0.509  }
         \end{enumerate}
         
     \end{enumerate}

    \item Suppose $X$ is a random variable with support $supp(X) = \{-2,-1,0,1,2,3\}$ and $Y = |X|$. Further assume the c.m.f of $X$ is 
    \begin{align}
        F_{X}(x) \begin{cases}
                     0.05 & \text{ if } x=-2\\
                     0.15 & \text{ if } x=-1\\
                     0.35 & \text{ if } x=0\\
                     0.65 & \text{ if } x=1\\
                     0.95 & \text{ if } x=2\\
                     1.00 & \text{ if } x=3
                  \end{cases}
    \end{align}
    \begin{enumerate}
            \item Define the support of $Y$
        \item Build the p.m.f of $Y$
        \item Build the c.m.f of $Y$
    \end{enumerate}
    
    \item Compute $\sum_{i=-5}^{i=5} i^{2}/2$
    
    \item Simplify the $V(X)$ into the equation $E(X^{2}) - \left[E(X)\right]^{2}$. Hint: Write down the definition of variance using the expectation, then expand the squared terms and simplify.
    
    \item Define the random variable $A$ with pmf
    \begin{align}
        f_{B}(b) = \begin{cases}
                      0.52 &\text{ if } b=0\\
                      0.12 &\text{ if } b=1\\
                      0.34 &\text{ if } b=2
                   \end{cases}
    \end{align} 
    \begin{enumerate}
        \item Compute $\mathbb{E}(B)$
        \item Compute $V(B)$
        \item Use Chebychev's inequality to make a statement about $P(B \geq b)$ 
    \end{enumerate}
    
\end{document}