\documentclass[krantz1,ChapterTOCs]{krantz}
\usepackage{fixltx2e,fix-cm}
\usepackage{amssymb}
\usepackage{amsmath}
\usepackage{graphicx}
\usepackage{subfigure}
\usepackage{makeidx}
\usepackage{multicol}
\usepackage{hyperref}
\usepackage{xcolor}

\begin{document}

\section{Exercises-Answer key}

\begin{enumerate}
    \item Suppose we collect the following dataset $\mathval{D} = (10,2,6,7,17,3,1,1,6,5,4,1)$ and further we assume that each data point $d_{i}$ was generated by sampling from a sequence of i.i.d random variables  $d_{1} \sim X_{1}, d_{2} \sim X_{2}, \cdots, d_{12} \sim X_{12}$ where $X_{i} \sim \text{Geom}(p)$.  
    \begin{enumerate}
        \item State the LLN for the above problem
        \begin{enumerate}
            \item { \color{red}  The LLN states that $\overline{X_{n}} \to \frac{1}{p}$    } 
        \end{enumerate}
        \item Compute the sample mean of $\mathval{D}$
        \begin{enumerate}
            \item { \color{red} 5.3 } 
        \end{enumerate}
        \item What can we say about the sample mean, using the LLN, if we collect 13,14,$\cdots$ data points?
        \begin{enumerate}
            \item { \color{red} That the sample average will get closer to the expected value } 
        \end{enumerate}
        
    \end{enumerate}
    
    \item  Suppose we collect the following sample $\mathcal{D} = (10,2,6,7,17,3,1,1,6,5,4,1)$ and further we assume that each data point $d_{i}$ was generated by sampling from a sequence of i.i.d random variables  $d_{1} \sim X_{1}, d_{2} \sim X_{2}, \cdots, d_{12} \sim X_{12}$ where $X_{i} \sim \text{Pois}(\lambda)$.
    \begin{enumerate}
        \item State the LLN for the above problem
        \begin{enumerate}
            \item { \color{red}  The LLN states that $\overline{X_{n}} \to \lambda$    } 
        \end{enumerate}
        
        \item Compute the sample variance
        \begin{enumerate}
            \item { \color{red} V(X) = 5.3  } 
        \end{enumerate}
 
        \item What can we say about the sample variance, using the LLN, if we collect 13,14,$\cdots$ data points?
        \begin{enumerate}
            \item { \color{red} That the sample variance will get closer to the true variance } 
        \end{enumerate}
    \end{enumerate}
    
    \item Define independent and identically distributed random variables $B_{1},B_{2},B_{3},B_{4},B_{5} \sim \text{Geom}(p)$. Suppose we sample these rvs and collect $d = (12,9,1,5,3)$.
    \begin{enumerate}
        \item Use the method of moments to develop an estimator for $p$.
        \begin{enumerate}
            \item { \color{red} 
            \begin{align}
              \overline{X}_{n}  &= \frac{1}{p} \\
              p &= \frac{1}{\overline{X}_{n}} 
            \end{align}
             } 
        \end{enumerate}
        
        \item Estimate $p$ from the above sample $\mathcal{D}$.
        \begin{enumerate}
            \item {\color{red} $p = \frac{5}{30} = 1/6 $  }
        \end{enumerate}
        
        \item What is the estimated expected value? 
        \begin{enumerate}
            \item {\color{red} $\hat{\mathbb{E}(X)} = \frac{1}{1/6} = 6  $
        \end{enumerate}
        
        
        \item What is the estimated variance? 
        \begin{enumerate}
            \item {\color{red} $\hat{V(X)} = \frac{1}{\hat{p}} \cdot \frac{1-\hat{p}}{\hat{p}} = 6 \cdot \frac{5/6}{1/6} = 30   $   }
        \end{enumerate}
        
        \item What is the estimated $P(B_{1} = 1)$
        \begin{enumerate}
            \item { \color{red} $ \hat{f(x)} = \hat{p} (1-\hat{p})^{x-1} = \hat{p} = 1/6$  }
        \end{enumerate}
        
    \end{enumerate}
    \item Assume $(Y_{1}, Y_{2},Y_{3}, \cdots, Y_{n})$ are a random sample where $Y_{i} \sim \mathcal{N}(\mu,\sigma^{2})$. Further assume we collected the data set $\mathcal{D} = (1.30, -2.11, 1.44, 0.35, -0.40, 0.61,-1.06, -0.86, -0.60, 0.19)$.
    \begin{enumerate}
        \item Use the method of moments to develop an estimator for $\mu$ and for $\sigma^{2}$.
        \begin{enumerate}
            \item {\color{red}
                \begin{align}
                     \overline{X}_{n}    &= \hat{\mu} \\ 
                    \overline{X^{2}}_{n} &= \hat{\sigma^{2}} + \hat{\mu}^{2} 
                \end{align}
            }
        \end{enumerate}
        
        \item Estimate $\mu$ and $\sigma^{2}$ from the above sample $\mathcal{D}$. 
        \begin{enumerate}
            \item {\color{red}
                \begin{align}
                    -0.11 &= \hat{\mu} \\ 
                    1.11  &= \hat{\sigma^{2}} + (-0.11)^{2} \\ 
                    \\ 
                    -0.11 &= \hat{\mu} \\ 
                    1.12 = 1.11 - (-0.11)^{2}  &= \hat{\sigma^{2}} 
                \end{align}
            }
        \end{enumerate}
        
        
        \item What is the estimated expected value? 
        \begin{enumerate}
            \item {\color{red} $\hat{\mathbb{E}(X)} = \hat{\mu} = -0.11$ 
        \end{enumerate}
        
        \item What is the estimated variance? 
        \begin{enumerate}
            \item {\color{red} $\hat{V(X)} = \hat{\sigma^{2}} = 1.12$ 
        \end{enumerate}
        
        \item What is the estimated $P(Y_{1} = 2)$
        \begin{enumerate}
            \item {\color{red} $P(Y=2) = 0$ 
        \end{enumerate}
        
    \end{enumerate}
    \item Lets assume we decide to study the incubation period for the influenza virus. The incubation period is defined as the number of days between when the virus infects a host and when that host becomes symptomatic. We collect from 5 individuals the date they came in contact with someone who was infected with influenza and the date they themselves were symptomatic.
    
    \begin{align}
        \mathcal{D} 
        = \begin{bmatrix}
            \text{Date of contact} & \text{Date of Symptoms}\\
            02/20  & 02/21\\
            04/12  & 04/17\\
            03/22  & 03/25\\
            10/24  & 10/26\\
            07/15  & 07/18\\
          \end{bmatrix}
    \end{align}
    
    \begin{enumerate}
        \item Provide a statistical setup for this data. Define a sample, assumptions about the sample, and a distribution for each random variable that is a part of the sample
        
        \begin{enumerate}
            \item {
            \color{red} 
            incubation\_times = $[1,5,3,2,3]$
            $(T_{1},T_{2},T_{3},T_{4},T_{5})$
            $T_{i} \sim \text{Geom}(p)$ or 
            $T_{i} \sim \text{Poisson}(\lambda)$ or 
            $T_{i} \sim \mathcal{N}(\mu,\sigma^{2})$ 
            } 
        \end{enumerate}
        
        \item Estimate the parameters in your statistical setup using the Method of Moments
        \begin{enumerate}
            \item {
            \color{red} 
                For Geom:    $p = \frac{1}{\overline{x}} = 1/ 2.8 = 0.36 $\\
                For Poisson: $\lambda = \overline{x} = 2.8 $\\
                For Normal:    $\mu = \overline{x} = 2.8; \sigma^{2} = 1.76   $\\
                
            } 
        \end{enumerate}
        
        
        \item Describe your results
        \begin{enumerate}
            \item {
            \color{red} 
                Should mention the parameter and characterize incubation times. Be geenrous. 
            } 
        \end{enumerate}

        
    \end{enumerate}
    
    \item Suppose $(Y_{1},Y_{2},\cdots,Y_{n})$ is a random sample where $Y_{i} \sim \text{Pois}(\lambda)$.
    \begin{enumerate}
        \item Define an estimator for $\lambda$ using the MoM
        \begin{enumerate}
            \item {
            \color{red} 
                $\hat{\lambda} = \overline{X}$
            } 
        \end{enumerate}
        
        \item Characterize the distribution of $\hat{\lambda}$ using the CLT.
        \begin{enumerate}
            \item {
            \color{red} 
                $\hat{\lambda} \sim \mathcal{N}( \lambda, \frac{\lambda}{n}  )
            } 
        \end{enumerate}

    \end{enumerate}
    
    \item The Aedes mosquito is a vector for yellow and dengue fever, the Zika virus, and Chikungunya. Tracking the daily incidence of Aedes mosquitoes in a given location is one signal associated with the incidence of these four diseases. Mosquitoes are routinely captured and counted and the daily frequency of mosquitoes is reported to departments of public health.
    
    Suppose we capture and count the number of mosquitoes in a specific county for two weeks, and compile this data in the following dataset $\mathcal{D} = (124,98,188,212,100,34,90,99,46,176,67,94,344,67)$ where one data point represents the number of mosquitoes collected in one day.
    
    \begin{enumerate}
        \item Define a set of random variables that we will use to model the daily frequency of mosquitoes. Include notation for the random sample and define a common Poisson distribution for all random variables.
        \begin{enumerate}
            \item {
            \color{red} 
            \begin{align}
                (C_{1}, C_{2}, C_{3}, \cdots, C_{14} )\\
                C_{i} \sim \text{Pois}(\lambda)
            \end{align}
            } 
        \end{enumerate}


        \item Use the method of moments to estimate parameter value $(\lambda)$ from the given dataset $\mathcal{D}$
        \begin{enumerate}
            \item {
            \color{red} 
                $\hat{\lambda} = \Bar{c} = 124.21$
            } 
        \end{enumerate}

        \item Estimate the probability that we observe 124 mosquitoes
        \begin{enumerate}
            \item {
            \color{red} 
                $\hat{f} = e^{-124.21} \frac{124.21^{124}}{124!}$
            } 
        \end{enumerate}

        \item Estimate the probability that we observe 120 to 125 mosquitoes
        \begin{enumerate}
            \item {
            \color{red} 
                \begin{align}
                    &P(C=120) + P(C=121) +P(C=122) +P(C=123) +P(C=124) + P(C=125) \\ 
                    & e^{-124.21} \left(\frac{124.21^{120}}{120!} + \frac{124.21^{121}}{121!} + \frac{124.21^{122}}{122!} + \frac{124.21^{123}}{123!} + \frac{124.21^{124}}{124!} + \frac{124.21^{125}}{125!} \right)    
                \end{align}
            } 
        \end{enumerate}

        \item Estimate the expected value and the variance
        \begin{enumerate}
            \item {
            \color{red} 
                    $\hat{\mathbb{E}(C)} = 124.21;\hat{V(C)} = 124.21 $
            } 
        \end{enumerate}

        \item Let's rework the above model and assume that each random variable follows a Normal distribution. Use the method of moments to estimate parameters value $(\mu, \sigma^{2})$ from the given dataset $\mathcal{D}$.
        \begin{enumerate}
            \item {
            \color{red} 
         
            \begin{align}
                \hat{\mu} = 124.21 \\ 
                \hat{\sigma^{2}} = 6254 \text{(within a few digits is ok)} \\ 
            \end{align}
        
            } 
        \end{enumerate}

        \item Estimate the expected value and the variance for the Normal model
        \begin{enumerate}
            \item {
            \color{red} 
        
                \begin{align}
                    \hat{\mathbb{E}(C)} = \hat{\mu} = 124.21 \\ 
                    \hat{V(C)} = \hat{\sigma^{2}}   = 6254 
                \end{align}
        
            } 
        \end{enumerate}

    \end{enumerate}
    
    \item A Randomized Control trial~(RCT) is a common study design to compare the efficacy and safety of a novel device. Suppose the sponsor for a new cardiovascular device wishes to compare the safety of their device, which they define as the percent of patients who survive 7 days after the procedure, and the efficacy, which they define as the percent of patients who survive at one year after the procedure. \\ 
    
    The trial enrolls 50 patients to receive the device~(the device group) and 50 patients who receive optimal medical therapy~(the control group). The trial enrolls these 100 patients over the course of two years. At year 3 all 100 patients have been contacted at one year after their procedure~(called a patient's one year followup).\\
    
    We find that in the device group 8 patients did not survive at or after 7 days from the date of their procedure and 18 patients did not survive one year after the date of their procedure. In the control group 9 patients did not survive at or after 7 days and 33 patients did not survive one year after the procedure.
    
    \begin{enumerate}
        \item Define a set of random variables that we will use to model the number of patients in the device group who survive 7 days after the procedure out of a total of 50 patients and a second set of random variables to model the number of patients in the control group who survive 7 days after the procedure out of a total of 50 patients. Include mathematical notation for the random sample and a common distribution for both sets of random variables. 
        \begin{enumerate}
            \item {
            \color{red} 
                \begin{align}
                    (D_{1},D_{2},\cdots, D_{50}) \\
                    D_{i} \sim \text{Bern}(\theta_{\text{device}})\\
                    \nonumber\\
                    (C_{1},C_{2},\cdots, C_{50}) \\
                    C_{i} \sim \text{Bern}(\theta_{\text{control}})
                \end{align}
            } 
        \end{enumerate}

        
        \item Use the Method of Moments to estimate the parameters for the distribution you chose for survival at or after 7 day for patients in the device group. Report the parameter estimate. 
        \begin{enumerate}
            \item {
            \color{red} 
        
            \begin{align}
                \text{Device, 7 days Survival} =  $ 42/50= 0.84$
            \end{align}
        
            } 
        \end{enumerate}

        
        \item Use the Method of Moments to estimate the parameters for the distribution you chose for survival at or after 7 days for patients in the control group. Report the parameter estimate.
        \begin{enumerate}
            \item {
            \color{red} 
            \begin{align}
                \text{Control, 7 days Survival} =  $ 41/50= 0.82$
            \end{align}
        
            } 
        \end{enumerate}

        
        \item What can you conclude about the safety of the novel device compared to the safety of optimal medical therapy?  
        \begin{enumerate}
            \item {
            \color{red} 
                The Device is as safe as the Control treatment
            } 
        \end{enumerate}

        
        \item Use the Method of Moments to estimate the parameters for the distribution you chose for survival at one year for patients in the device group. Report the parameter estimate.
        \begin{enumerate}
            \item {
            \color{red} 
            
            \begin{align}
                \text{Device, 365 days Survival} =  $ 32/50= 0.64$
            \end{align}
        
            } 
        \end{enumerate}

        
        \item Use the Method of Moments to estimate the parameters for the distribution you chose for survival at one year for patients in the control group. Report the parameter estimate. 
        \begin{enumerate}
            \item {
            \color{red} 
            
            \begin{align}
                \text{Control, 365 days Survival} =  $ 17/50= 0.34$
            \end{align}
        
            } 
        \end{enumerate}

        
        \item What can you conclude about the efficacy of the novel device compared to the efficacy of optimal medical therapy?   
        \begin{enumerate}
            \item {
            \color{red} 
                Device is more efficacious than control
            } 
        \end{enumerate}
    
    
    \end{enumerate}
    
    \item Suppose a random sample $(Y_{1},Y_{2},\cdots,Y_{10})$ generates the following dataset $\mathcal{D} = ( 0,-2.3,0.4,9.2,10,-9.8,3,3,3,0)$. Assume that $X_{i} \sim \mathcal{N}(\mu,\sigma^{2})$ and compute the corresponding ten z-scores.
        \begin{enumerate}
            \item {
            \color{red} 
                mean = 1.65
                sd = 5.34
                $z_{i} = (x_{i} - 1.65)/(5.34)$
                \begin{table}[ht!]
                    \centering
                    \begin{tabular}{c|c}
                    x & z\\
                    \hline
                    0      & -0.31  \\
                    -2.3   & -0.74 \\
                    0.4    & -0.23\\
                    9.2     & 1.41 \\
                    10      & 1.56\\
                    -9.8    &-2.14\\
                    3       &0.25\\
                    3       &0.25\\
                    3       &0.25\\
                    0       &-0.31\\
                    \end{tabular}
                    \caption{Caption}
                    \label{tab:my_label}
                \end{table}
            
            } 
        \end{enumerate}

    \clearpage
    \item Suppose we collect the following dataset $\mathcal{D} = ( 6.51, -1.11,  7.29,  0.23,  3.45,  0.85, -0.42,  5.66,  0.04, -1.31 )$. Further, lets assume $X_{i} \sim \mathcal{N}(\mu,\sigma^{2})$.
    \begin{enumerate}
        \item Please compute a $1-\alpha$ confidence interval for $\mu$. The symbol $z_{1-\alpha/2}$ should appear in this interval. 
        \begin{enumerate}
            \item {
            \color{red} 
        
            } 
        \end{enumerate}

        \item Please compute a 95\% confidence interval for $\mu$.
        \begin{enumerate}
            \item {
            \color{red} 
        
            } 
        \end{enumerate}

        \item Please compute a 80\% confidence interval for $\mu$.
        \begin{enumerate}
            \item {
            \color{red} 
        
            } 
        \end{enumerate}

        \item Why is your 95\% confidence interval larger than your 80\% confidence interval?
        \begin{enumerate}
            \item {
            \color{red} 
        
            } 
        \end{enumerate}

    \end{enumerate}
    
    \item Suppose we collect the following dataset $\mathcal{D} = (4, 0,  1, 2, 8, 13, 0, 1, 0, 3)$. Further, lets assume $X_{i} \sim \text{Geom}(p)$.
        \begin{enumerate}
            \item Please compute a $1-\alpha$ confidence interval for $\frac{1}{p}$. The symbol $z_{1-\alpha/2}$ should appear in this interval. 
            \begin{enumerate}
                 \item {
                     \color{red} 
        
                     } 
            \end{enumerate}

            \item Please compute a 95\% confidence interval for $\frac{1}{p}$.
        \begin{enumerate}
            \item {
            \color{red} 
        
            } 
        \end{enumerate}

            \item Please compute a 80\% confidence interval for $\frac{1}{p}$.
        \begin{enumerate}
            \item {
            \color{red} 
        
            } 
        \end{enumerate}

        \end{enumerate}
        
    \item Suppose we collect the following dataset $\mathcal{D} = (0, 1, 1, 1, 1, 1, 1, 0, 1, 1)$. Further, lets assume $X_{i} \sim \text{Bernoulli}(\theta)$.
        \begin{enumerate}
            \item Please compute a $1-\alpha$ confidence interval for $\theta$. The symbol $z_{1-\alpha/2}$ should appear in this interval. 
        \begin{enumerate}
            \item {
            \color{red} 
        
            } 
        \end{enumerate}

            \item Please compute a 95\% confidence interval for $\theta$.
        \begin{enumerate}
            \item {
            \color{red} 
        
            } 
        \end{enumerate}

            \item Please compute a 80\% confidence interval for $\theta$.
        \begin{enumerate}
            \item {
            \color{red} 
        
            } 
        \end{enumerate}

        \end{enumerate}
        
    \item Suppose we collect the following dataset $\mathcal{D} = (6, 6, 2, 7, 3, 3, 5, 5, 7, 5)$. Further, lets assume $X_{i} \sim \text{Binomial}(50,\theta)$.
        \begin{enumerate}
            \item Please compute a $1-\alpha$ confidence interval for $\theta$. The symbol $z_{1-\alpha/2}$ should appear in this interval. 
            \begin{enumerate}
            \item {
            \color{red} 
        
            } 
        \end{enumerate}

            \item Please compute a 95\% confidence interval for $\theta$.
            \begin{enumerate}
            \item {
            \color{red} 
        
            } 
        \end{enumerate}

            \item Please compute a 80\% confidence interval for $\theta$.
            \begin{enumerate}
            \item {
            \color{red} 
        
            } 
        \end{enumerate}

        \end{enumerate}
        

    \item Suppose we collect the following dataset $\mathcal{D} = (1, 4, 2, 1, 1, 4, 3, 2, 3, 0)$. Further, lets assume $X_{i} \sim \text{Poisson}(\lambda)$.
        \begin{enumerate}
            \item Please compute a $1-\alpha$ confidence interval for $\lambda$. The symbol $z_{1-\alpha/2}$ should appear in this interval. 
            \begin{enumerate}
            \item {
            \color{red} 
        
            } 
        \end{enumerate}

            \item Please compute a 95\% confidence interval for $\lambda$.
        \begin{enumerate}
            \item {
            \color{red} 
        
            } 
        \end{enumerate}

            \item Please compute a 80\% confidence interval for $\lambda$.
                \begin{enumerate}
            \item {
            \color{red} 
        
            } 
        \end{enumerate}

        \end{enumerate}

\end{enumerate}


\end{document}