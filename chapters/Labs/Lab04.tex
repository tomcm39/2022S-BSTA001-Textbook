
\chapterauthor{thomas mcandrew, david braun}{Lehigh University}
%\chapterauthor{Second Author}{Second Author Affiliation}
\chapter{Laboratory 04}


\hypertarget{matrix-algebra}{%
\section{Matrix Algebra}\label{matrix-algebra}}

Statisticians and data scientists use the language of vectors and
matrices to organize data and work with models. We will take a close
look at matrix algebra, taking time to relate matrix algebra to the more
familar algebra that you have worked with in the past.

    \hypertarget{recap-on-vectors-and-operations-on-vectors}{%
\subsection{Recap on vectors and operations on
vectors}\label{recap-on-vectors-and-operations-on-vectors}}

\hypertarget{definition}{%
\subsubsection{Definition}\label{definition}}

A vector \(v\) is an ordered list of real numbers. We denote a vector
with a lower case letter and enclose the values in the vector with
square brackets. We can write a vector \(v\) as a \emph{row vector}

\begin{align}
    v = [1,2,3,4,5]
\end{align}

or as a \emph{column vector}

\begin{align}
    v = \left [
        \begin{matrix}
            1 \\ 
            2 \\
            3 \\
            4 \\
            5 \\
        \end{matrix} \right ]
\end{align}.

A vector has a \textbf{length}, defined as the number of values included
in that vector. For example, the above vector is length 5.

We can create a vector in R by using the c operator.

    \begin{tcolorbox}[breakable, size=fbox, boxrule=1pt, pad at break*=1mm,colback=cellbackground, colframe=cellborder]
\prompt{In}{incolor}{1}{\boxspacing}
\begin{Verbatim}[commandchars=\\\{\}]
\PY{n}{v} \PY{o}{=}\PY{n+nf}{c}\PY{p}{(}\PY{l+m}{1}\PY{p}{,}\PY{l+m}{2}\PY{p}{,}\PY{l+m}{3}\PY{p}{,}\PY{l+m}{4}\PY{p}{,}\PY{l+m}{5}\PY{p}{)}
\end{Verbatim}
\end{tcolorbox}

    \hypertarget{vector-times-a-scalar}{%
\subsubsection{Vector times a scalar}\label{vector-times-a-scalar}}

A vector of length one is called a \textbf{scalar}. We can define the
results of multiplying a vector \(v=[1,2,3,4]\) by a scalar \(\alpha\)
as each entry of the vector times the scalar \(\alpha\).

\begin{align}
    \alpha v = \left [
                    \begin{matrix}
                    \alpha \times 1 \\ 
                    \alpha \times 2 \\
                    \alpha \times 3 \\
                    \alpha \times 4 \\
                \end{matrix} \right ]
\end{align}

For example, the vector \(v = [1,2,3,4,5]\) times the scalar
\(\alpha=7\) will result in a vector

\begin{align}
    \alpha v &= [7 \times 1, 7 \times 2, 7 \times 3, 7 \times 4  ]\\
             &= [7, 14, 21, 28]
\end{align}

R understands how to multiply vectors and scalars with no additional
syntax.

    \begin{tcolorbox}[breakable, size=fbox, boxrule=1pt, pad at break*=1mm,colback=cellbackground, colframe=cellborder]
\prompt{In}{incolor}{3}{\boxspacing}
\begin{Verbatim}[commandchars=\\\{\}]
\PY{n}{v} \PY{o}{=} \PY{n+nf}{c}\PY{p}{(}\PY{l+m}{1}\PY{p}{,}\PY{l+m}{2}\PY{p}{,}\PY{l+m}{3}\PY{p}{,}\PY{l+m}{4}\PY{p}{)}
\PY{n}{alpha} \PY{o}{=} \PY{l+m}{7}

\PY{n}{alpha}\PY{o}{*}\PY{n}{v}
\end{Verbatim}
\end{tcolorbox}

    \begin{enumerate*}
\item 7
\item 14
\item 21
\item 28
\end{enumerate*}


    
    \hypertarget{vector-plusminus-a-vector}{%
\subsubsection{Vector plus/minus a
vector}\label{vector-plusminus-a-vector}}

A vector \(v\) plus a vector \(q\) creates a new vector that adds the
individual entries of \(v\) and \(q\)

\begin{align}
    v &= [4,-1,7] \\ 
    q &= [0,32,9] \\ 
    v + q &= [ 4+0, -1+32, 7+9] = [4,31,16]
\end{align}

A vector \(v\) minus a vector \(q\) creates a new vector that subtracts
the individual entries of \(q\) from \(v\)

\begin{align}
    v &= [4,-1,7] \\ 
    q &= [0,32,9] \\ 
    v - q &= [ 4-0, -1-32, 7-9] = [4,-33,-2]\\
    q - v &= [ 0-4, 32-(-1), 9-7] = [-4,33,2]
\end{align}

R also understands vector addition and subtraction with no special
syntax.

    \begin{tcolorbox}[breakable, size=fbox, boxrule=1pt, pad at break*=1mm,colback=cellbackground, colframe=cellborder]
\prompt{In}{incolor}{6}{\boxspacing}
\begin{Verbatim}[commandchars=\\\{\}]
\PY{n}{v} \PY{o}{=} \PY{n+nf}{c}\PY{p}{(}\PY{l+m}{4}\PY{p}{,} \PY{l+m}{\PYZhy{}1}\PY{p}{,} \PY{l+m}{7}\PY{p}{)}
\PY{n}{q} \PY{o}{=} \PY{n+nf}{c}\PY{p}{(}\PY{l+m}{0}\PY{p}{,} \PY{l+m}{32}\PY{p}{,} \PY{l+m}{9}\PY{p}{)}

\PY{n}{add} \PY{o}{=} \PY{n}{v}\PY{o}{+}\PY{n}{q}
\PY{n}{subtract} \PY{o}{=} \PY{n}{v}\PY{o}{\PYZhy{}}\PY{n}{q}

\PY{n+nf}{print}\PY{p}{(}\PY{n}{add}\PY{p}{)}
\PY{n+nf}{print}\PY{p}{(}\PY{n}{subtract}\PY{p}{)}
\end{Verbatim}
\end{tcolorbox}

    \begin{Verbatim}[commandchars=\\\{\}]
[1]  4 31 16
[1]   4 -33  -2
    \end{Verbatim}

    \hypertarget{the-matrix}{%
\subsection{The Matrix}\label{the-matrix}}

A \textbf{matrix} is an ordered list of vectors.

\begin{align}
    M = \left [ \begin{matrix}
            1 & 2 & 3 \\
            4 & 5 & 6 \\
            3 & 2 & 13 \\
            90 & 23 & 0 \\
         \end{matrix} \right ]   
\end{align}

A matrix has a dimension which is an ordered pair of numbers: the first
number indicates the number of rows of the matrix and the second number
indicates the number of columns. For example, he matrix \(M\) above has
dimension \((4,3)\).

To build a matrix in R we can issue the matrix command. Matrix is a
function that takes a vector as an argument and can take one of many
optional arguments.

If we give the matrix function only a vector, the default behavior of
this function is to create a vector.

    \begin{tcolorbox}[breakable, size=fbox, boxrule=1pt, pad at break*=1mm,colback=cellbackground, colframe=cellborder]
\prompt{In}{incolor}{8}{\boxspacing}
\begin{Verbatim}[commandchars=\\\{\}]
\PY{n}{A} \PY{o}{=} \PY{n+nf}{matrix}\PY{p}{(}\PY{n+nf}{c}\PY{p}{(}\PY{l+m}{1}\PY{p}{,}\PY{l+m}{2}\PY{p}{,}\PY{l+m}{3}\PY{p}{,}\PY{l+m}{4}\PY{p}{,}\PY{l+m}{5}\PY{p}{,}\PY{l+m}{6}\PY{p}{)}\PY{p}{)}
\PY{n}{A}
\end{Verbatim}
\end{tcolorbox}

    A matrix: 6 × 1 of type dbl
\begin{tabular}{l}
	 1\\
	 2\\
	 3\\
	 4\\
	 5\\
	 6\\
\end{tabular}


    
    This is because a vector can be thought of as a matrix with one column.
In other words, a matrix with dimension \((N,1)\) is a vector of length
\(N\).

We can provide another arguement to the \texttt{matrix}
function---ncol---to specify the number of columns we want to create.

    \begin{tcolorbox}[breakable, size=fbox, boxrule=1pt, pad at break*=1mm,colback=cellbackground, colframe=cellborder]
\prompt{In}{incolor}{9}{\boxspacing}
\begin{Verbatim}[commandchars=\\\{\}]
\PY{n}{A} \PY{o}{=} \PY{n+nf}{matrix}\PY{p}{(}\PY{n+nf}{c}\PY{p}{(}\PY{l+m}{1}\PY{p}{,}\PY{l+m}{2}\PY{p}{,}\PY{l+m}{3}\PY{p}{,}\PY{l+m}{4}\PY{p}{,}\PY{l+m}{5}\PY{p}{,}\PY{l+m}{6}\PY{p}{)}\PY{p}{,} \PY{n}{ncol}\PY{o}{=}\PY{l+m}{2}\PY{p}{)}
\PY{n}{A}
\end{Verbatim}
\end{tcolorbox}

    A matrix: 3 × 2 of type dbl
\begin{tabular}{ll}
	 1 & 4\\
	 2 & 5\\
	 3 & 6\\
\end{tabular}


    
    Above we used the ncol argument to specify a matrix with two columns. R
will automatically determine the number of rows for the matrix and
fill-in the values of the matrix column by column.

If we want, we can ask R to fill in values of the matrix row by row by
including an additional argument in the matrix function\texttt{byrow}

    \begin{tcolorbox}[breakable, size=fbox, boxrule=1pt, pad at break*=1mm,colback=cellbackground, colframe=cellborder]
\prompt{In}{incolor}{10}{\boxspacing}
\begin{Verbatim}[commandchars=\\\{\}]
\PY{n}{v} \PY{o}{=} \PY{n+nf}{c}\PY{p}{(}\PY{l+m}{1}\PY{p}{,}\PY{l+m}{2}\PY{p}{,}\PY{l+m}{3}\PY{p}{,}\PY{l+m}{4}\PY{p}{,}\PY{l+m}{5}\PY{p}{,}\PY{l+m}{6}\PY{p}{)}
\PY{n}{A} \PY{o}{=} \PY{n+nf}{matrix}\PY{p}{(}\PY{n}{v}\PY{p}{,} \PY{n}{ncol}\PY{o}{=}\PY{l+m}{2}\PY{p}{)}

\PY{n}{B} \PY{o}{=} \PY{n+nf}{matrix}\PY{p}{(}\PY{n}{v}\PY{p}{,} \PY{n}{ncol}\PY{o}{=}\PY{l+m}{2}\PY{p}{,} \PY{n}{byrow}\PY{o}{=}\PY{k+kc}{TRUE}\PY{p}{)}

\PY{n+nf}{print}\PY{p}{(}\PY{n}{A}\PY{p}{)}
\PY{n+nf}{print}\PY{p}{(}\PY{n}{B}\PY{p}{)}
\end{Verbatim}
\end{tcolorbox}

    \begin{Verbatim}[commandchars=\\\{\}]
     [,1] [,2]
[1,]    1    4
[2,]    2    5
[3,]    3    6
     [,1] [,2]
[1,]    1    2
[2,]    3    4
[3,]    5    6
    \end{Verbatim}

    We see that matrix A was ``column filled'' and matrix B was ``row
filled''.

    There are similar operations for matrices as there are for vectors.

    \hypertarget{matrix-plusminus-a-matrix}{%
\subsubsection{Matrix plus/minus a
matrix}\label{matrix-plusminus-a-matrix}}

Given a matrix \(A\) and matrix \(B\), the sum of \(A\) and \(B\) is a
new matrix \(C\) where the i,j entry of \(C\) (\(C_{ij}\)) is the sum of
the corresponding entries from \(A\) and \(B\)
(\(C_{ij} = A_{ij} + B_{ij}\)). To add two matrices they must have the
same dimension.

\begin{align}
    A &= \left [ \begin{matrix}
                    1 & 2 & 3 \\ 
                    4 & 5 & 6 \\
                \end{matrix}
        \right ] \\ 
    B &=  \left [ \begin{matrix}
                    6 & 5 & 4 \\ 
                    3 & 2 & 1 \\
                \end{matrix}
        \right ] \\
    %
    C &= A + B = \left [ \begin{matrix}
                    1+6 & 2+5 & 3+4 \\ 
                    4+3 & 5+2 & 6+1 \\
                \end{matrix}
        \right ]  = \left [ \begin{matrix}
                    7 & 7 & 7 \\ 
                    7 & 7 & 7 \\
                \end{matrix}
        \right ] \\ 
    D &= A - B = \left [ \begin{matrix}
                    1-6 & 2-5 & 3-4 \\ 
                    4-3 & 5-2 & 6-1 \\
                \end{matrix}
        \right ]  = \left [ \begin{matrix}
                    -5 & -3 & -1 \\ 
                    1 & 3 & 5 \\
                \end{matrix}
        \right ] \\     
\end{align}

R understands matrix addition and subtraction without any additional
syntax.

    \begin{tcolorbox}[breakable, size=fbox, boxrule=1pt, pad at break*=1mm,colback=cellbackground, colframe=cellborder]
\prompt{In}{incolor}{17}{\boxspacing}
\begin{Verbatim}[commandchars=\\\{\}]
\PY{n}{A} \PY{o}{=} \PY{n+nf}{matrix}\PY{p}{(}\PY{n+nf}{c}\PY{p}{(}\PY{l+m}{\PYZhy{}1}\PY{p}{,}\PY{l+m}{2}\PY{p}{,}\PY{l+m}{4}\PY{p}{,}\PY{l+m}{1}\PY{p}{,}\PY{l+m}{3}\PY{p}{,}\PY{l+m}{8}\PY{p}{)}\PY{p}{,} \PY{n}{ncol}\PY{o}{=}\PY{l+m}{3}\PY{p}{)}
\PY{n}{D} \PY{o}{=} \PY{n+nf}{matrix}\PY{p}{(}\PY{n+nf}{c}\PY{p}{(}\PY{l+m}{90}\PY{p}{,}\PY{l+m}{0.34}\PY{p}{,}\PY{l+m}{1.54}\PY{p}{,}\PY{l+m}{\PYZhy{}9.43}\PY{p}{,}\PY{l+m}{10}\PY{p}{,}\PY{l+m}{0}\PY{p}{)}\PY{p}{,} \PY{n}{ncol}\PY{o}{=}\PY{l+m}{3}\PY{p}{)}

\PY{n+nf}{print}\PY{p}{(}\PY{n}{A}\PY{p}{)}
\PY{n+nf}{print}\PY{p}{(}\PY{n}{D}\PY{p}{)}

\PY{n}{A}\PY{o}{+}\PY{n}{D}
\end{Verbatim}
\end{tcolorbox}

    \begin{Verbatim}[commandchars=\\\{\}]
     [,1] [,2] [,3]
[1,]   -1    4    3
[2,]    2    1    8
      [,1]  [,2] [,3]
[1,] 90.00  1.54   10
[2,]  0.34 -9.43    0
    \end{Verbatim}

    A matrix: 2 × 3 of type dbl
\begin{tabular}{lll}
	 89.00 &  5.54 & 13\\
	  2.34 & -8.43 &  8\\
\end{tabular}


    
    \begin{tcolorbox}[breakable, size=fbox, boxrule=1pt, pad at break*=1mm,colback=cellbackground, colframe=cellborder]
\prompt{In}{incolor}{14}{\boxspacing}
\begin{Verbatim}[commandchars=\\\{\}]
\PY{n}{A}\PY{o}{\PYZhy{}}\PY{n}{D}
\end{Verbatim}
\end{tcolorbox}

    A matrix: 2 × 3 of type dbl
\begin{tabular}{lll}
	 -91.00 &  2.46 & -7\\
	   1.66 & 10.43 &  8\\
\end{tabular}


    
    \hypertarget{matrix-times-a-matrix}{%
\subsubsection{Matrix times a Matrix}\label{matrix-times-a-matrix}}

Two matrices \(A\) and \(B\) can be multiplied together \(C = AB\) if
the number of columns of \(A\) is equal to the number of rows of \(B\).
In other words, if the dimension of A is (a,b) and the dimension of B is
(c,d) then b and c must be equal.

The i,j entry of this product, \(C_{ij}\) is the following sum of
products:

\begin{align}
    C_{i,j} &= a_{i,1}b_{1,j} + a_{i,2}b_{2,j} + a_{i,3}b_{3,j} + \cdots + a_{i,N}b_{N,j}\\
            &= \sum_{e=1}^{N} a_{i,e}b_{e,j}
\end{align}

For example, suppose that \begin{align}
    A = \left [ \begin{matrix}
                     1 & 2\\
                     3 & 4 \\
                 \end{matrix} \right] 
\end{align} and that \begin{align}
    B = \left [ \begin{matrix}
                     1 & 2 & -1\\
                     3 & 4 & -2 \\
                 \end{matrix} \right] . 
\end{align} Then the product \(C = AB\) equals

\begin{align}
    C = AB = \left [ \begin{matrix}
                            1*1 + 2*3 & 1*2 + 2*4 & 1*-1 + 2*-2 \\
                            3*1 + 4*3 & 3*2 + 4*4 & 3*-1 + 4*-2 \\
                      \end{matrix} \right] = 
                    \left [ \begin{matrix}
                            7 & 10 & -5 \\
                            15 & 22 & -11 \\
                      \end{matrix} \right] 
\end{align}

The above definition and computation can feel cumbersome. We can
simplify the above calculations by introducing the inner product.

The inner product between two \textbf{vectors} \(v\) and \(q\), or
\(v'q\), is

\begin{align}
    v &= [1,2,3]\\
    q &= [4,5,6]\\
    \\
    v'q &= 1 \cdot 4 + 2 \cdot 5 + 3 \cdot 6 \\
        &= 4 + 10 + 18 = 32
\end{align}

Let us use the inner product to simplify the above matrix
multiplication. First, we rewrite the matrix A as a stack of two
\emph{row} vectors

\begin{align}
    A = \begin{bmatrix}
         a_{1} \\ 
         a_{2} \\
        \end{bmatrix}
\end{align}

where \(a_{1} = [1 , 2]\) and \(a_{2} = [3,4]\)

Second, we rewrite the matrix B as a stack of three \emph{column}
vectors

\begin{align}
    B = \begin{bmatrix}
         b_{1} & b_{2} & b_{3}
        \end{bmatrix}
\end{align}

where \$b\_\{1\} =

\begin{bmatrix} 1 \\ 3 \end{bmatrix}

\$, \$b\_\{2\} =

\begin{bmatrix} 2 \\ 4 \end{bmatrix}

\$, and \$b\_\{3\} =

\begin{bmatrix} -1 \\ -2 \end{bmatrix}

\$

Then the product AB is a matrix of inner products

\begin{align}
    C = \begin{bmatrix}
             a_{1}'b_{1} & a_{1}'b_{2} & a_{1}'b_{3} \\ 
             a_{2}'b_{1} & a_{2}'b_{2} & a_{2}'b_{3} \\ 
         \end{bmatrix}
\end{align}

Matrix multiplication is different than mulitplication between two
variables that represent real numbers because matrix multiplication is
\textbf{not} commutative---the product \(AB\) is not rarely equal to
\(BA\).

To compute the product of two matrices in R we need the \%*\% operator

    \begin{tcolorbox}[breakable, size=fbox, boxrule=1pt, pad at break*=1mm,colback=cellbackground, colframe=cellborder]
\prompt{In}{incolor}{22}{\boxspacing}
\begin{Verbatim}[commandchars=\\\{\}]
\PY{n}{A} \PY{o}{=} \PY{n+nf}{matrix}\PY{p}{(}\PY{n+nf}{c}\PY{p}{(}\PY{l+m}{\PYZhy{}1}\PY{p}{,}\PY{l+m}{0}\PY{p}{,}\PY{l+m}{1}\PY{p}{,}\PY{l+m}{\PYZhy{}2}\PY{p}{,}\PY{l+m}{\PYZhy{}1}\PY{p}{,}\PY{l+m}{3}\PY{p}{,}\PY{l+m}{4}\PY{p}{,}\PY{l+m}{5}\PY{p}{,}\PY{l+m}{6}\PY{p}{)}\PY{p}{,}\PY{n}{ncol}\PY{o}{=}\PY{l+m}{3}\PY{p}{)}
\PY{n}{B} \PY{o}{=} \PY{n+nf}{matrix}\PY{p}{(}\PY{n+nf}{c}\PY{p}{(}\PY{l+m}{1}\PY{p}{,}\PY{l+m}{2}\PY{p}{,}\PY{l+m}{3}\PY{p}{,}\PY{l+m}{4}\PY{p}{,}\PY{l+m}{5}\PY{p}{,}\PY{l+m}{6}\PY{p}{,}\PY{l+m}{7}\PY{p}{,}\PY{l+m}{8}\PY{p}{,}\PY{l+m}{9}\PY{p}{)}\PY{p}{,}\PY{n}{ncol}\PY{o}{=}\PY{l+m}{3}\PY{p}{)}

\PY{n+nf}{print}\PY{p}{(}\PY{n}{A}\PY{p}{)}
\PY{n+nf}{print}\PY{p}{(}\PY{n}{B}\PY{p}{)}

\PY{n+nf}{print}\PY{p}{(}\PY{n}{A}\PY{o}{\PYZpc{}*\PYZpc{}}\PY{n}{B}\PY{p}{)}
\PY{n+nf}{print}\PY{p}{(}\PY{n}{B}\PY{o}{\PYZpc{}*\PYZpc{}}\PY{n}{A}\PY{p}{)}
\end{Verbatim}
\end{tcolorbox}

    \begin{Verbatim}[commandchars=\\\{\}]
     [,1] [,2] [,3]
[1,]   -1   -2    4
[2,]    0   -1    5
[3,]    1    3    6
     [,1] [,2] [,3]
[1,]    1    4    7
[2,]    2    5    8
[3,]    3    6    9
     [,1] [,2] [,3]
[1,]    7   10   13
[2,]   13   25   37
[3,]   25   55   85
     [,1] [,2] [,3]
[1,]    6   15   66
[2,]    6   15   81
[3,]    6   15   96
    \end{Verbatim}

    Note above that we computed the product AB and BA and these products
resulted in different matrices.

    \hypertarget{matrix-times-a-vector}{%
\subsubsection{Matrix times a vector}\label{matrix-times-a-vector}}

A matrix \(A\) with dimension \((r,c)\) can be multiplied by a vector
\(v\) if the length of \(v\) is \(c\). Then the product \(Av\) is a
vector with the the i\(^{\text{th}}\) entry of \(Av\) defined as
\begin{align}
    (Av)_{i} = \sum_{k=1}^{c} A_{i,k} v_{k}
\end{align}

For example, Let the matrix \begin{align}
    A = \left [ \begin{matrix}
                   1 & 2 \\ 
                   3 & 4 \\
                   5 & 6 \\
         \end{matrix} \right ]
\end{align} and the vector \begin{align}
    v = \left [ \begin{matrix}
                   4  \\ 
                   -3  \\
         \end{matrix} \right ]
\end{align} then

\begin{align}
    Av = \left [ \begin{matrix}
                   1*4+2*-3 \\ 
                   3*4+4*-3 \\
                   5*4+6*-3 \\
         \end{matrix} \right ] = 
         \left [ \begin{matrix}
                   -2  \\ 
                   0  \\
                   2 \\
         \end{matrix} \right ]
\end{align}

The definition above can also be simplifed by using inner products. Let

\begin{align}
    A = \begin{bmatrix}
          a_{1} \\ 
          a_{2} \\ 
          a_{3} 
        \end{bmatrix}
\end{align}

where \(a_{1} = [1,2]\), \(a_{2} = [3,4]\), and \(a_{3} = [5,6]\). Then
the product \(Av\) simplifies to

\begin{align}
    Av = \begin{bmatrix}
            a_{1}'v\\
            a_{2}'v\\
            a_{3}'v
         \end{bmatrix}
\end{align}

We can use the \%*\% operator in R to multiply together a matrix and a
vector

    \begin{tcolorbox}[breakable, size=fbox, boxrule=1pt, pad at break*=1mm,colback=cellbackground, colframe=cellborder]
\prompt{In}{incolor}{32}{\boxspacing}
\begin{Verbatim}[commandchars=\\\{\}]
\PY{n}{A} \PY{o}{=} \PY{n+nf}{matrix}\PY{p}{(}\PY{n+nf}{c}\PY{p}{(}\PY{l+m}{1}\PY{p}{,}\PY{l+m}{2}\PY{p}{,}\PY{l+m}{3}\PY{p}{,}\PY{l+m}{4}\PY{p}{,}\PY{l+m}{5}\PY{p}{,}\PY{l+m}{6}\PY{p}{)}\PY{p}{,} \PY{n}{ncol}\PY{o}{=}\PY{l+m}{2}\PY{p}{,} \PY{n}{byrow}\PY{o}{=}\PY{k+kc}{TRUE}\PY{p}{)}
\PY{n}{v} \PY{o}{=} \PY{n+nf}{c}\PY{p}{(}\PY{l+m}{4}\PY{p}{,}\PY{l+m}{\PYZhy{}3}\PY{p}{)}

\PY{n+nf}{print}\PY{p}{(}\PY{l+s}{\PYZdq{}}\PY{l+s}{Matrix A\PYZdq{}}\PY{p}{)}
\PY{n+nf}{print}\PY{p}{(}\PY{n}{A}\PY{p}{)}

\PY{n+nf}{print}\PY{p}{(}\PY{l+s}{\PYZdq{}}\PY{l+s}{Vector v\PYZdq{}}\PY{p}{)}
\PY{n+nf}{print}\PY{p}{(}\PY{n}{v}\PY{p}{)}

\PY{n+nf}{print}\PY{p}{(}\PY{l+s}{\PYZdq{}}\PY{l+s}{Av\PYZdq{}}\PY{p}{)}
\PY{n}{product} \PY{o}{=} \PY{n}{A}\PY{o}{\PYZpc{}*\PYZpc{}}\PY{n}{v}
\PY{n+nf}{print}\PY{p}{(}\PY{n}{product}\PY{p}{)}
\end{Verbatim}
\end{tcolorbox}

    \begin{Verbatim}[commandchars=\\\{\}]
[1] "Matrix A"
     [,1] [,2]
[1,]    1    2
[2,]    3    4
[3,]    5    6
[1] "Vector v"
[1]  4 -3
[1] "Av"
     [,1]
[1,]   -2
[2,]    0
[3,]    2
    \end{Verbatim}

    \hypertarget{matrix-transpose}{%
\subsubsection{Matrix Transpose}\label{matrix-transpose}}

Given a matrix \(M\), the \textbf{tranpose} of \(M\), \(M^{'}\) or
\(M^{\text{T}}\), is a matrix where the first row of \(M^{'}\)
corresponds to the first column of \(M\), the second row of \(M^{'}\)
corresponds to the second columns of \(M\) and so on. If \(M\) has
dimension \((r,c)\) than \(M^{'}\) has dimensions \((c,r)\).

For example, if \begin{align}
    M = \left [ \begin{matrix}
            1 & 2 & 3 \\
            4 & 5 & 6 \\
            3 & 2 & 13 \\
            90 & 23 & 0 \\
         \end{matrix} \right ]
\end{align} then the tranpose of \(M\), \begin{align}
    M^{'} = \left [ \begin{matrix}
            1 & 4 & 3 & 90 \\
            2 & 5 & 2 & 23 \\ 
            3 &  6 &  13 & 0 \\
         \end{matrix} \right ]   
\end{align}

We can use the \texttt{t()} operator to transpose a matrix in R.

    \begin{tcolorbox}[breakable, size=fbox, boxrule=1pt, pad at break*=1mm,colback=cellbackground, colframe=cellborder]
\prompt{In}{incolor}{36}{\boxspacing}
\begin{Verbatim}[commandchars=\\\{\}]
\PY{n}{M} \PY{o}{=} \PY{n+nf}{matrix}\PY{p}{(}\PY{n+nf}{c}\PY{p}{(} \PY{l+m}{1}\PY{p}{,}\PY{l+m}{2}\PY{p}{,}\PY{l+m}{3}\PY{p}{,}\PY{l+m}{4}\PY{p}{,}\PY{l+m}{5}\PY{p}{,}\PY{l+m}{6}\PY{p}{,}\PY{l+m}{3}\PY{p}{,}\PY{l+m}{2}\PY{p}{,}\PY{l+m}{13}\PY{p}{,}\PY{l+m}{90}\PY{p}{,}\PY{l+m}{23}\PY{p}{,}\PY{l+m}{0}\PY{p}{)}\PY{p}{,}\PY{n}{ncol}\PY{o}{=}\PY{l+m}{3}\PY{p}{,}\PY{n}{byrow}\PY{o}{=}\PY{k+kc}{TRUE}\PY{p}{)}

\PY{n+nf}{print}\PY{p}{(}\PY{l+s}{\PYZdq{}}\PY{l+s}{The matrix M\PYZdq{}}\PY{p}{)}
\PY{n+nf}{print}\PY{p}{(}\PY{n}{M}\PY{p}{)}

\PY{n+nf}{print}\PY{p}{(}\PY{l+s}{\PYZdq{}}\PY{l+s}{The transpose\PYZdq{}}\PY{p}{)}
\PY{n}{Mt} \PY{o}{=} \PY{n+nf}{t}\PY{p}{(}\PY{n}{M}\PY{p}{)}

\PY{n+nf}{print}\PY{p}{(}\PY{n}{Mt}\PY{p}{)}
\end{Verbatim}
\end{tcolorbox}

    \begin{Verbatim}[commandchars=\\\{\}]
[1] "The matrix M"
     [,1] [,2] [,3]
[1,]    1    2    3
[2,]    4    5    6
[3,]    3    2   13
[4,]   90   23    0
[1] "The transpose"
     [,1] [,2] [,3] [,4]
[1,]    1    4    3   90
[2,]    2    5    2   23
[3,]    3    6   13    0
    \end{Verbatim}

    \hypertarget{matrix-inverse}{%
\subsubsection{Matrix inverse}\label{matrix-inverse}}

For a super fun overview of matrix multiplication and the concept of
identities and inverses, click
\href{https://www.youtube.com/watch?v=vTtx-mC61Tw}{here}!

\hypertarget{the-identity-matrix}{%
\paragraph{The identity matrix}\label{the-identity-matrix}}

The \textbf{identity matrix} of dimension \(r\), usually labled
\(I_{r}\), is a matrix with \(r\) rows and \(r\) columns such that the
diagonal elements of the matrix, the (1,1) entry, (2,2) entry, up to
(r,r) entry are the number 1 and all other entries are 0. For example,

\begin{align}
    I_{2} = \left [ \begin{matrix}
                        1 & 0 \\ 
                        0 & 1 \\
                     \end{matrix}
             \right ] 
\end{align} This matrix is called the identity matrix because any matrix
\(A\) times \(I\) returns \(A\). To be more precise, this matrix is the
\emph{multiplicative} identity matrix. But the adjective
\emph{multiplicative} is usually dropped.

\hypertarget{the-inverse-matrix}{%
\paragraph{The inverse matrix}\label{the-inverse-matrix}}

The \textbf{inverse} of a matrix \(A\)---\(A^{-1}\)---is a matrix such
that, if \(A\) has the same number of rows and columns (called square)
then \begin{align}
    A A^{-1} = A^{-1}A = I
\end{align}

The idea of a matrix inverse is the same as in algebra. If we have a
variable \(a\) then the inverse of \(a\), called \(a^{-1}\) is the
unique number such that \[ a \cdot a^{-1} = a^{-1} \cdot a = 1. \]

In matrix algebra, the identity matrix takes the place of the ``1'' in
algebra.

We can compute the inverse of a matrix \(A\) in R using the
\texttt{solve} function.

    \begin{tcolorbox}[breakable, size=fbox, boxrule=1pt, pad at break*=1mm,colback=cellbackground, colframe=cellborder]
\prompt{In}{incolor}{41}{\boxspacing}
\begin{Verbatim}[commandchars=\\\{\}]
\PY{n}{A} \PY{o}{=} \PY{n+nf}{matrix}\PY{p}{(}\PY{n+nf}{c}\PY{p}{(}\PY{l+m}{3}\PY{p}{,}\PY{l+m}{4}\PY{p}{,}\PY{l+m}{5}\PY{p}{,}\PY{l+m}{6}\PY{p}{)}\PY{p}{,}\PY{n}{ncol}\PY{o}{=}\PY{l+m}{2}\PY{p}{)}

\PY{n+nf}{print}\PY{p}{(}\PY{l+s}{\PYZdq{}}\PY{l+s}{A\PYZdq{}}\PY{p}{)}
\PY{n+nf}{print}\PY{p}{(}\PY{n}{A}\PY{p}{)}

\PY{n+nf}{print}\PY{p}{(}\PY{l+s}{\PYZdq{}}\PY{l+s}{A inverse\PYZdq{}}\PY{p}{)}
\PY{n}{Ainverse} \PY{o}{=} \PY{n+nf}{solve}\PY{p}{(}\PY{n}{A}\PY{p}{)}
\PY{n+nf}{print}\PY{p}{(}\PY{n}{Ainverse}\PY{p}{)}
\end{Verbatim}
\end{tcolorbox}

    \begin{Verbatim}[commandchars=\\\{\}]
[1] "A"
     [,1] [,2]
[1,]    3    5
[2,]    4    6
[1] "A inverse"
     [,1] [,2]
[1,]   -3  2.5
[2,]    2 -1.5
    \end{Verbatim}

    \hypertarget{assignment}{%
\subsection{Assignment}\label{assignment}}

\begin{enumerate}
\def\labelenumi{\arabic{enumi}.}
\tightlist
\item
  Define the vector v of length 3 with the following numbers: -1,0,1
\item
  Define the matrix A with dimension (2,3) by filling, column wise, the
  values: 1,2,3,4,5,6
\item
  Multiply A by v (Av)
\item
  Multiply v by A (vA). Why does R return an error?
\item
  Write a function that takes two argument which are both matrices and
  returns the product of these matrices.
\end{enumerate}

