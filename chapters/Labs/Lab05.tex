\chapterauthor{thomas mcandrew, david braun}{Lehigh University}
%\chapterauthor{Second Author}{Second Author Affiliation}
\chapter{Laboratory 05}

\hypertarget{data-frame}{%
\subsection{Data Frame}\label{data-frame}}

We learned in lecture that a dataframe is a specific way to organize
data points sampled from a sample space \(\mathcal{G}\). A dataframe
supposes a dataset \(\mathcal{D} = ( d_{1},d_{2},\cdots,d_{N})\) should
be organized so that each row is a single outcome \(d_{i}\) and each
column is a position in the tuple \(d_{i} = ( x,y,z,... )\)

The dataframe is a common way computational scientists think about data.

    \hypertarget{how-to-read-a-csv-file-into-a-data-frame}{%
\subsubsection{How to read a CSV file into a data
frame}\label{how-to-read-a-csv-file-into-a-data-frame}}

The function \texttt{read.csv} takes as an argument a string that
indicates a file on your local computer OR a URL online. Below the
\texttt{read.csv} function is used to import into memory a
\textbf{dataframe} related to bad drivers from the statsitcal news
outlet called \emph{FiveThirtyEight}.

The news article is here
https://fivethirtyeight.com/features/which-state-has-the-worst-drivers/

    \begin{tcolorbox}[breakable, size=fbox, boxrule=1pt, pad at break*=1mm,colback=cellbackground, colframe=cellborder]
\prompt{In}{incolor}{23}{\boxspacing}
\begin{Verbatim}[commandchars=\\\{\}]
\PY{n}{d} \PY{o}{=} \PY{n+nf}{read.csv}\PY{p}{(}\PY{l+s}{\PYZdq{}}\PY{l+s}{https://raw.githubusercontent.com/fivethirtyeight/data/master/bad\PYZhy{}drivers/bad\PYZhy{}drivers.csv\PYZdq{}}\PY{p}{)}
\PY{n}{d} \PY{c+c1}{\PYZsh{} Jupyter will by default print any variable}
\end{Verbatim}
\end{tcolorbox}

    A data.frame: 51 × 8
\begin{tabular}{llllllll}
 State & Number.of.drivers.involved.in.fatal.collisions.per.billion.miles & Percentage.Of.Drivers.Involved.In.Fatal.Collisions.Who.Were.Speeding & Percentage.Of.Drivers.Involved.In.Fatal.Collisions.Who.Were.Alcohol.Impaired & Percentage.Of.Drivers.Involved.In.Fatal.Collisions.Who.Were.Not.Distracted & Percentage.Of.Drivers.Involved.In.Fatal.Collisions.Who.Had.Not.Been.Involved.In.Any.Previous.Accidents & Car.Insurance.Premiums.... & Losses.incurred.by.insurance.companies.for.collisions.per.insured.driver....\\
 <chr> & <dbl> & <int> & <int> & <int> & <int> & <dbl> & <dbl>\\
\hline
	 Alabama              & 18.8 & 39 & 30 &  96 &  80 &  784.55 & 145.08\\
	 Alaska               & 18.1 & 41 & 25 &  90 &  94 & 1053.48 & 133.93\\
	 Arizona              & 18.6 & 35 & 28 &  84 &  96 &  899.47 & 110.35\\
	 Arkansas             & 22.4 & 18 & 26 &  94 &  95 &  827.34 & 142.39\\
	 California           & 12.0 & 35 & 28 &  91 &  89 &  878.41 & 165.63\\
	 Colorado             & 13.6 & 37 & 28 &  79 &  95 &  835.50 & 139.91\\
	 Connecticut          & 10.8 & 46 & 36 &  87 &  82 & 1068.73 & 167.02\\
	 Delaware             & 16.2 & 38 & 30 &  87 &  99 & 1137.87 & 151.48\\
	 District of Columbia &  5.9 & 34 & 27 & 100 & 100 & 1273.89 & 136.05\\
	 Florida              & 17.9 & 21 & 29 &  92 &  94 & 1160.13 & 144.18\\
	 Georgia              & 15.6 & 19 & 25 &  95 &  93 &  913.15 & 142.80\\
	 Hawaii               & 17.5 & 54 & 41 &  82 &  87 &  861.18 & 120.92\\
	 Idaho                & 15.3 & 36 & 29 &  85 &  98 &  641.96 &  82.75\\
	 Illinois             & 12.8 & 36 & 34 &  94 &  96 &  803.11 & 139.15\\
	 Indiana              & 14.5 & 25 & 29 &  95 &  95 &  710.46 & 108.92\\
	 Iowa                 & 15.7 & 17 & 25 &  97 &  87 &  649.06 & 114.47\\
	 Kansas               & 17.8 & 27 & 24 &  77 &  85 &  780.45 & 133.80\\
	 Kentucky             & 21.4 & 19 & 23 &  78 &  76 &  872.51 & 137.13\\
	 Louisiana            & 20.5 & 35 & 33 &  73 &  98 & 1281.55 & 194.78\\
	 Maine                & 15.1 & 38 & 30 &  87 &  84 &  661.88 &  96.57\\
	 Maryland             & 12.5 & 34 & 32 &  71 &  99 & 1048.78 & 192.70\\
	 Massachusetts        &  8.2 & 23 & 35 &  87 &  80 & 1011.14 & 135.63\\
	 Michigan             & 14.1 & 24 & 28 &  95 &  77 & 1110.61 & 152.26\\
	 Minnesota            &  9.6 & 23 & 29 &  88 &  88 &  777.18 & 133.35\\
	 Mississippi          & 17.6 & 15 & 31 &  10 & 100 &  896.07 & 155.77\\
	 Missouri             & 16.1 & 43 & 34 &  92 &  84 &  790.32 & 144.45\\
	 Montana              & 21.4 & 39 & 44 &  84 &  85 &  816.21 &  85.15\\
	 Nebraska             & 14.9 & 13 & 35 &  93 &  90 &  732.28 & 114.82\\
	 Nevada               & 14.7 & 37 & 32 &  95 &  99 & 1029.87 & 138.71\\
	 New Hampshire        & 11.6 & 35 & 30 &  87 &  83 &  746.54 & 120.21\\
	 New Jersey           & 11.2 & 16 & 28 &  86 &  78 & 1301.52 & 159.85\\
	 New Mexico           & 18.4 & 19 & 27 &  67 &  98 &  869.85 & 120.75\\
	 New York             & 12.3 & 32 & 29 &  88 &  80 & 1234.31 & 150.01\\
	 North Carolina       & 16.8 & 39 & 31 &  94 &  81 &  708.24 & 127.82\\
	 North Dakota         & 23.9 & 23 & 42 &  99 &  86 &  688.75 & 109.72\\
	 Ohio                 & 14.1 & 28 & 34 &  99 &  82 &  697.73 & 133.52\\
	 Oklahoma             & 19.9 & 32 & 29 &  92 &  94 &  881.51 & 178.86\\
	 Oregon               & 12.8 & 33 & 26 &  67 &  90 &  804.71 & 104.61\\
	 Pennsylvania         & 18.2 & 50 & 31 &  96 &  88 &  905.99 & 153.86\\
	 Rhode Island         & 11.1 & 34 & 38 &  92 &  79 & 1148.99 & 148.58\\
	 South Carolina       & 23.9 & 38 & 41 &  96 &  81 &  858.97 & 116.29\\
	 South Dakota         & 19.4 & 31 & 33 &  98 &  86 &  669.31 &  96.87\\
	 Tennessee            & 19.5 & 21 & 29 &  82 &  81 &  767.91 & 155.57\\
	 Texas                & 19.4 & 40 & 38 &  91 &  87 & 1004.75 & 156.83\\
	 Utah                 & 11.3 & 43 & 16 &  88 &  96 &  809.38 & 109.48\\
	 Vermont              & 13.6 & 30 & 30 &  96 &  95 &  716.20 & 109.61\\
	 Virginia             & 12.7 & 19 & 27 &  87 &  88 &  768.95 & 153.72\\
	 Washington           & 10.6 & 42 & 33 &  82 &  86 &  890.03 & 111.62\\
	 West Virginia        & 23.8 & 34 & 28 &  97 &  87 &  992.61 & 152.56\\
	 Wisconsin            & 13.8 & 36 & 33 &  39 &  84 &  670.31 & 106.62\\
	 Wyoming              & 17.4 & 42 & 32 &  81 &  90 &  791.14 & 122.04\\
\end{tabular}


    
    \hypertarget{selecting-a-column-and-the-operator}{%
\subsubsection{Selecting a column and the \$
operator}\label{selecting-a-column-and-the-operator}}

With a dataframe you can select rows by asking R to return the number
column where the first column in the dataframe is 1, the second column
2, and so on. You can also ask R to return a column of your dataframe by
column name.

The 3rd column of the dataframe that we read records, by state, the
percentage of fatal collision that involved a driver who was impaired by
alcohol.

We can ask R for the 3rd column by using square brackets in the same way
that we used square brackets to select rows and columns of matrices.

    \begin{tcolorbox}[breakable, size=fbox, boxrule=1pt, pad at break*=1mm,colback=cellbackground, colframe=cellborder]
\prompt{In}{incolor}{24}{\boxspacing}
\begin{Verbatim}[commandchars=\\\{\}]
\PY{n}{d}\PY{p}{[}\PY{p}{,}\PY{l+m}{3}\PY{p}{]}
\end{Verbatim}
\end{tcolorbox}

    \begin{enumerate*}
\item 39
\item 41
\item 35
\item 18
\item 35
\item 37
\item 46
\item 38
\item 34
\item 21
\item 19
\item 54
\item 36
\item 36
\item 25
\item 17
\item 27
\item 19
\item 35
\item 38
\item 34
\item 23
\item 24
\item 23
\item 15
\item 43
\item 39
\item 13
\item 37
\item 35
\item 16
\item 19
\item 32
\item 39
\item 23
\item 28
\item 32
\item 33
\item 50
\item 34
\item 38
\item 31
\item 21
\item 40
\item 43
\item 30
\item 19
\item 42
\item 34
\item 36
\item 42
\end{enumerate*}


    
    We can request a column from a dataframe by asking for 1. The dataframe
2. square brackets Inside the square brackets you divide the rows you
want selected and the columns you want selected by a comma.

For example, if we wanted to look at the 10th row and 3rd column we can
write

    \begin{tcolorbox}[breakable, size=fbox, boxrule=1pt, pad at break*=1mm,colback=cellbackground, colframe=cellborder]
\prompt{In}{incolor}{25}{\boxspacing}
\begin{Verbatim}[commandchars=\\\{\}]
\PY{n}{d}\PY{p}{[}\PY{l+m}{10}\PY{p}{,}\PY{l+m}{3}\PY{p}{]}
\end{Verbatim}
\end{tcolorbox}

    21

    
    If instead we wanted to view rows 10,11,12 and columns 3 and 4 we could
write

    \begin{tcolorbox}[breakable, size=fbox, boxrule=1pt, pad at break*=1mm,colback=cellbackground, colframe=cellborder]
\prompt{In}{incolor}{26}{\boxspacing}
\begin{Verbatim}[commandchars=\\\{\}]
\PY{n}{d}\PY{p}{[} \PY{l+m}{10}\PY{o}{:}\PY{l+m}{12}\PY{p}{,} \PY{n+nf}{c}\PY{p}{(}\PY{l+m}{3}\PY{p}{,}\PY{l+m}{4}\PY{p}{)}  \PY{p}{]}
\end{Verbatim}
\end{tcolorbox}

    A data.frame: 3 × 2
\begin{tabular}{r|ll}
  & Percentage.Of.Drivers.Involved.In.Fatal.Collisions.Who.Were.Speeding & Percentage.Of.Drivers.Involved.In.Fatal.Collisions.Who.Were.Alcohol.Impaired\\
  & <int> & <int>\\
\hline
	10 & 21 & 29\\
	11 & 19 & 25\\
	12 & 54 & 41\\
\end{tabular}


    
    The above examples select columns and rows by number. One advantage of a
dataframe is that we can select columns \textbf{by name}.

For example, we may be interested in rows 10-12 of the column
``Percentage.Of.Drivers.Involved.In.Fatal.Collisions.Who.Were.Speeding''

    \begin{tcolorbox}[breakable, size=fbox, boxrule=1pt, pad at break*=1mm,colback=cellbackground, colframe=cellborder]
\prompt{In}{incolor}{27}{\boxspacing}
\begin{Verbatim}[commandchars=\\\{\}]
\PY{n}{d}\PY{p}{[}\PY{l+m}{10}\PY{o}{:}\PY{l+m}{12}\PY{p}{,}\PY{l+s}{\PYZdq{}}\PY{l+s}{Percentage.Of.Drivers.Involved.In.Fatal.Collisions.Who.Were.Speeding\PYZdq{}}\PY{p}{]}
\end{Verbatim}
\end{tcolorbox}

    \begin{enumerate*}
\item 21
\item 19
\item 54
\end{enumerate*}


    
    We can ask for \textbf{all} rows that correspond to a column by leaving
the entry to the left of the comma blank.

    \begin{tcolorbox}[breakable, size=fbox, boxrule=1pt, pad at break*=1mm,colback=cellbackground, colframe=cellborder]
\prompt{In}{incolor}{28}{\boxspacing}
\begin{Verbatim}[commandchars=\\\{\}]
\PY{n}{d}\PY{p}{[}\PY{p}{,}\PY{l+s}{\PYZdq{}}\PY{l+s}{Percentage.Of.Drivers.Involved.In.Fatal.Collisions.Who.Were.Speeding\PYZdq{}}\PY{p}{]}
\end{Verbatim}
\end{tcolorbox}

    \begin{enumerate*}
\item 39
\item 41
\item 35
\item 18
\item 35
\item 37
\item 46
\item 38
\item 34
\item 21
\item 19
\item 54
\item 36
\item 36
\item 25
\item 17
\item 27
\item 19
\item 35
\item 38
\item 34
\item 23
\item 24
\item 23
\item 15
\item 43
\item 39
\item 13
\item 37
\item 35
\item 16
\item 19
\item 32
\item 39
\item 23
\item 28
\item 32
\item 33
\item 50
\item 34
\item 38
\item 31
\item 21
\item 40
\item 43
\item 30
\item 19
\item 42
\item 34
\item 36
\item 42
\end{enumerate*}


    
    Finally, there is a shorthand for the code
\texttt{d{[},"Percentage.Of.Drivers.Involved.In.Fatal.Collisions.Who.Were.Speeding"{]}}
using the dollarsign operator. The dollar sign operator can be thought
of as a function that takes a column name as input and returns the
column of data inside your data frame corresponding to that column.

    \begin{tcolorbox}[breakable, size=fbox, boxrule=1pt, pad at break*=1mm,colback=cellbackground, colframe=cellborder]
\prompt{In}{incolor}{41}{\boxspacing}
\begin{Verbatim}[commandchars=\\\{\}]
\PY{n}{d}\PY{o}{\PYZdl{}}\PY{n}{Percentage.Of.Drivers.Involved.In.Fatal.Collisions.Who.Were.Speeding}
\end{Verbatim}
\end{tcolorbox}

    
    \begin{Verbatim}[commandchars=\\\{\}]
NULL
    \end{Verbatim}

    
    \hypertarget{logical-indexing}{%
\subsubsection{Logical indexing}\label{logical-indexing}}

We are allowed to use logical indexing like we learned when selecting
items in vectors and matrices to select rows and columns of a data
frame. For example, suppose we are interested in \textbf{all} columns of
our data frame where the Percentage Of Drivers Involved In Fatal
Collisions Who Were Speeding is above 40\%.

    \begin{tcolorbox}[breakable, size=fbox, boxrule=1pt, pad at break*=1mm,colback=cellbackground, colframe=cellborder]
\prompt{In}{incolor}{30}{\boxspacing}
\begin{Verbatim}[commandchars=\\\{\}]
\PY{n}{d}\PY{p}{[} \PY{n}{d}\PY{o}{\PYZdl{}} \PY{l+s}{\PYZdq{}}\PY{l+s}{Percentage.Of.Drivers.Involved.In.Fatal.Collisions.Who.Were.Speeding\PYZdq{}} \PY{o}{\PYZgt{}} \PY{l+m}{40}\PY{p}{,} \PY{p}{]}
\end{Verbatim}
\end{tcolorbox}

    A data.frame: 8 × 8
\begin{tabular}{r|llllllll}
  & State & Number.of.drivers.involved.in.fatal.collisions.per.billion.miles & Percentage.Of.Drivers.Involved.In.Fatal.Collisions.Who.Were.Speeding & Percentage.Of.Drivers.Involved.In.Fatal.Collisions.Who.Were.Alcohol.Impaired & Percentage.Of.Drivers.Involved.In.Fatal.Collisions.Who.Were.Not.Distracted & Percentage.Of.Drivers.Involved.In.Fatal.Collisions.Who.Had.Not.Been.Involved.In.Any.Previous.Accidents & Car.Insurance.Premiums.... & Losses.incurred.by.insurance.companies.for.collisions.per.insured.driver....\\
  & <chr> & <dbl> & <int> & <int> & <int> & <int> & <dbl> & <dbl>\\
\hline
	2 & Alaska       & 18.1 & 41 & 25 & 90 & 94 & 1053.48 & 133.93\\
	7 & Connecticut  & 10.8 & 46 & 36 & 87 & 82 & 1068.73 & 167.02\\
	12 & Hawaii       & 17.5 & 54 & 41 & 82 & 87 &  861.18 & 120.92\\
	26 & Missouri     & 16.1 & 43 & 34 & 92 & 84 &  790.32 & 144.45\\
	39 & Pennsylvania & 18.2 & 50 & 31 & 96 & 88 &  905.99 & 153.86\\
	45 & Utah         & 11.3 & 43 & 16 & 88 & 96 &  809.38 & 109.48\\
	48 & Washington   & 10.6 & 42 & 33 & 82 & 86 &  890.03 & 111.62\\
	51 & Wyoming      & 17.4 & 42 & 32 & 81 & 90 &  791.14 & 122.04\\
\end{tabular}


    
    or maybe we are interested in not all columns, but just the states where
this percentage is above 40\%. We can subset to a specific column by
including in square brackets the name of the column

    \begin{tcolorbox}[breakable, size=fbox, boxrule=1pt, pad at break*=1mm,colback=cellbackground, colframe=cellborder]
\prompt{In}{incolor}{34}{\boxspacing}
\begin{Verbatim}[commandchars=\\\{\}]
\PY{n}{d}\PY{p}{[} \PY{n}{d}\PY{o}{\PYZdl{}} \PY{l+s}{\PYZdq{}}\PY{l+s}{Percentage.Of.Drivers.Involved.In.Fatal.Collisions.Who.Were.Speeding\PYZdq{}} \PY{o}{\PYZgt{}} \PY{l+m}{40}\PY{p}{,} \PY{l+s}{\PYZdq{}}\PY{l+s}{State\PYZdq{}} \PY{p}{]}
\end{Verbatim}
\end{tcolorbox}

    \begin{enumerate*}
\item 'Alaska'
\item 'Connecticut'
\item 'Hawaii'
\item 'Missouri'
\item 'Pennsylvania'
\item 'Utah'
\item 'Washington'
\item 'Wyoming'
\end{enumerate*}


    
    If we wanted to include more than one column then we can include a
vector that contains each column we want to select.

    \begin{tcolorbox}[breakable, size=fbox, boxrule=1pt, pad at break*=1mm,colback=cellbackground, colframe=cellborder]
\prompt{In}{incolor}{36}{\boxspacing}
\begin{Verbatim}[commandchars=\\\{\}]
\PY{n}{d}\PY{p}{[} \PY{n}{d}\PY{o}{\PYZdl{}}\PY{l+s}{\PYZdq{}}\PY{l+s}{Percentage.Of.Drivers.Involved.In.Fatal.Collisions.Who.Were.Speeding\PYZdq{}} \PY{o}{\PYZgt{}} \PY{l+m}{40}\PY{p}{,} \PY{n+nf}{c}\PY{p}{(}\PY{l+s}{\PYZdq{}}\PY{l+s}{State\PYZdq{}}\PY{p}{,}\PY{l+s}{\PYZdq{}}\PY{l+s}{Car.Insurance.Premiums....\PYZdq{}}\PY{p}{)} \PY{p}{]}
\end{Verbatim}
\end{tcolorbox}

    A data.frame: 8 × 2
\begin{tabular}{r|ll}
  & State & Car.Insurance.Premiums....\\
  & <chr> & <dbl>\\
\hline
	2 & Alaska       & 1053.48\\
	7 & Connecticut  & 1068.73\\
	12 & Hawaii       &  861.18\\
	26 & Missouri     &  790.32\\
	39 & Pennsylvania &  905.99\\
	45 & Utah         &  809.38\\
	48 & Washington   &  890.03\\
	51 & Wyoming      &  791.14\\
\end{tabular}


    
    \hypertarget{functions-for-data-frames}{%
\subsubsection{Functions for data
frames}\label{functions-for-data-frames}}

There are several useful functions that take as an argument a data
frame. The \texttt{nrow} function takes a data frame as an argument and
returns the number of rows (i.e.~the number of data points) in the data
frame. The \texttt{ncol} function takes a data frame as an argument and
returns the number of columns in the data frame.

    \begin{tcolorbox}[breakable, size=fbox, boxrule=1pt, pad at break*=1mm,colback=cellbackground, colframe=cellborder]
\prompt{In}{incolor}{31}{\boxspacing}
\begin{Verbatim}[commandchars=\\\{\}]
\PY{n}{number\PYZus{}of\PYZus{}rows} \PY{o}{=} \PY{n+nf}{nrow}\PY{p}{(}\PY{n}{d}\PY{p}{)}
\PY{n}{number\PYZus{}of\PYZus{}columns} \PY{o}{=} \PY{n+nf}{ncol}\PY{p}{(}\PY{n}{d}\PY{p}{)}

\PY{n+nf}{print}\PY{p}{(}\PY{n}{number\PYZus{}of\PYZus{}rows}\PY{p}{)}
\PY{n+nf}{print}\PY{p}{(}\PY{n}{number\PYZus{}of\PYZus{}columns}\PY{p}{)}
\end{Verbatim}
\end{tcolorbox}

    \begin{Verbatim}[commandchars=\\\{\}]
[1] 51
[1] 8
    \end{Verbatim}

    The \texttt{colnames} function takes as an argument a data frame and
returns a vector that contains the name of each column in the data
frame.

    \begin{tcolorbox}[breakable, size=fbox, boxrule=1pt, pad at break*=1mm,colback=cellbackground, colframe=cellborder]
\prompt{In}{incolor}{32}{\boxspacing}
\begin{Verbatim}[commandchars=\\\{\}]
\PY{n}{column\PYZus{}names} \PY{o}{=} \PY{n+nf}{colnames}\PY{p}{(}\PY{n}{d}\PY{p}{)}
\PY{n+nf}{print}\PY{p}{(}\PY{n}{column\PYZus{}names}\PY{p}{)}
\end{Verbatim}
\end{tcolorbox}

    \begin{Verbatim}[commandchars=\\\{\}]
[1] "State"
[2] "Number.of.drivers.involved.in.fatal.collisions.per.billion.miles"
[3] "Percentage.Of.Drivers.Involved.In.Fatal.Collisions.Who.Were.Speeding"
[4]
"Percentage.Of.Drivers.Involved.In.Fatal.Collisions.Who.Were.Alcohol.Impaired"
[5] "Percentage.Of.Drivers.Involved.In.Fatal.Collisions.Who.Were.Not.Distracted"
[6] "Percentage.Of.Drivers.Involved.In.Fatal.Collisions.Who.Had.Not.Been.Involve
d.In.Any.Previous.Accidents"
[7] "Car.Insurance.Premiums{\ldots}"
[8]
"Losses.incurred.by.insurance.companies.for.collisions.per.insured.driver{\ldots}"
    \end{Verbatim}

    The \texttt{summary} function is a function that takes as an argument a
data frame and returns, for each column in the data frame, the minimum
value, maximum value, mean (average), median, 25th percentile (called
the 1st quartile), and the 75th percentile (called the 3rd quartile).

    \begin{tcolorbox}[breakable, size=fbox, boxrule=1pt, pad at break*=1mm,colback=cellbackground, colframe=cellborder]
\prompt{In}{incolor}{33}{\boxspacing}
\begin{Verbatim}[commandchars=\\\{\}]
\PY{n+nf}{summary}\PY{p}{(}\PY{n}{d}\PY{p}{)}
\end{Verbatim}
\end{tcolorbox}

    
    \begin{Verbatim}[commandchars=\\\{\}]
    State          
 Length:51         
 Class :character  
 Mode  :character  
                   
                   
                   
 Number.of.drivers.involved.in.fatal.collisions.per.billion.miles
 Min.   : 5.90                                                   
 1st Qu.:12.75                                                   
 Median :15.60                                                   
 Mean   :15.79                                                   
 3rd Qu.:18.50                                                   
 Max.   :23.90                                                   
 Percentage.Of.Drivers.Involved.In.Fatal.Collisions.Who.Were.Speeding
 Min.   :13.00                                                       
 1st Qu.:23.00                                                       
 Median :34.00                                                       
 Mean   :31.73                                                       
 3rd Qu.:38.00                                                       
 Max.   :54.00                                                       
 Percentage.Of.Drivers.Involved.In.Fatal.Collisions.Who.Were.Alcohol.Impaired
 Min.   :16.00                                                               
 1st Qu.:28.00                                                               
 Median :30.00                                                               
 Mean   :30.69                                                               
 3rd Qu.:33.00                                                               
 Max.   :44.00                                                               
 Percentage.Of.Drivers.Involved.In.Fatal.Collisions.Who.Were.Not.Distracted
 Min.   : 10.00                                                            
 1st Qu.: 83.00                                                            
 Median : 88.00                                                            
 Mean   : 85.92                                                            
 3rd Qu.: 95.00                                                            
 Max.   :100.00                                                            
 Percentage.Of.Drivers.Involved.In.Fatal.Collisions.Who.Had.Not.Been.Involved.In.Any.Previous.Accidents
 Min.   : 76.00                                                                                        
 1st Qu.: 83.50                                                                                        
 Median : 88.00                                                                                        
 Mean   : 88.73                                                                                        
 3rd Qu.: 95.00                                                                                        
 Max.   :100.00                                                                                        
 Car.Insurance.Premiums{\ldots}
 Min.   : 642.0            
 1st Qu.: 768.4            
 Median : 859.0            
 Mean   : 887.0            
 3rd Qu.:1007.9            
 Max.   :1301.5            
 Losses.incurred.by.insurance.companies.for.collisions.per.insured.driver{\ldots}
 Min.   : 82.75                                                              
 1st Qu.:114.64                                                              
 Median :136.05                                                              
 Mean   :134.49                                                              
 3rd Qu.:151.87                                                              
 Max.   :194.78                                                              
    \end{Verbatim}

    
    \hypertarget{assignment}{%
\subsection{Assignment}\label{assignment}}

We are going to look at a dataset that was collected on University
``Fight Songs'' that are played during sporting events. The article
about the data set is here=
https://projects.fivethirtyeight.com/college-fight-song-lyrics/

\begin{enumerate}
\def\labelenumi{\arabic{enumi}.}
\tightlist
\item
  The data is at the URL
  (https://raw.githubusercontent.com/fivethirtyeight/data/master/fight-songs/fight-songs.csv).
  Please read in this raw CSV as a data.frame
\item
  Use the nrow and ncol function to describe the number of fight songs
  sampled and the number of different characteristics collected about
  each song.
\item
  How many songs were sampled with a BPM (beats per minute) above 150?
\item
  Select the rows where BPM is greater than 150 and select the column
  ``nonsense'' (Whether or not the song uses nonsense syllables
  (e.g.~``Whoo-Rah'' or ``Hooperay'') )
\item
  Summarize the data frame. Why do you think the summary function does
  not produce useful information for many of the columns?
\end{enumerate}
