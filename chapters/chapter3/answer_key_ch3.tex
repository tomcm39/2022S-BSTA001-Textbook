\documentclass[krantz1,ChapterTOCs]{krantz}
\usepackage{fixltx2e,fix-cm}
\usepackage{amssymb}
\usepackage{amsmath}
\usepackage{graphicx}
\usepackage{subfigure}
\usepackage{makeidx}
\usepackage{multicol}
\usepackage{hyperref}
\usepackage{xcolor}

\begin{document}

\section{Exercises-Answer key}

\begin{enumerate}
    \item Let $X \sim \text{Bernoulli}(0.2)$
    \begin{enumerate}
        \item P(X=0) = ?
        \begin{enumerate}
            \item {\color{red} $1-\theta = 0.8$  }
        \end{enumerate}
        \item P(X=1) = ?
        \begin{enumerate}
            \item {\color{red} $\theta = 0.2$  }
        \end{enumerate}

        \item Please compute $\mathbb{E}(X)$
        \begin{enumerate}
            \item {\color{red} $\mathbb{E}(X) = \theta = 0.2$  }
        \end{enumerate}

        \item Please compute $V(X)$
        \begin{enumerate}
            \item {\color{red} $V(X) = \theta (1-\theta) 0.2 \cdot 0.8 = 0.16$  }
        \end{enumerate}

        \item Define the $supp(X)$
        \begin{enumerate}
            \item {\color{red} $supp(X) = \{0,1\}$  }
        \end{enumerate}

    \end{enumerate}
    
    \item Let $Y \sim \text{Bernoulli}(\theta)$, and 
    show that $P(Y=1) = \mathbb{E}(Y)$
    \begin{enumerate}
        \item {\color{red} 
            \begin{align}
                P(Y=1) &= \theta\\
                \mathbb{E}(Y) = \theta\\
                P(Y=1) = \mathbb{E}(Y)
            \end{align}
          }
    \end{enumerate}


    \item Let $Y \sim \text{Bernoulli}(\theta)$, and 
    show that $V(Y) \leq \mathbb{E}(Y)$
    \begin{enumerate}
        \item {\color{red} 
            $V(Y) = \theta (1-\theta)$ and because $\theta \in [0,1]$ then $1-\theta \in [0,1]$ and so $\mathbb{E}(Y) = \theta > \theta (1-\theta)$
        }
    \end{enumerate}
    \item Let $Y \sim \text{Bernoulli}(\theta)$, and let $Z$ be the following function of $Y$:
    \begin{align}
        Z(y) = \begin{cases}
                1 & \text{ if } y = 0\\
                0 & \text{ if } y = 1
            \end{cases}
    \end{align}
    What probability distribution does $Z$ follow and why?
    \begin{enumerate}
        \item {\color{red} 
            \begin{align}
                supp(Z) = \{0,1\} \\ 
                P(Z = 1) = 1 - \theta \\ 
                P(Z =0) = \theta \\
                Z \sim \text{Bern}(1-\theta)
            \end{align}
        }
    \end{enumerate}
    
    
    \item Design an experiment (short description) and define a random variable $Y$ that may follow a geometric distribution. In the context of your experiment, how would you communicate $\mathbb{E}(Y)$ to another without statistical expertise?
    
    \item Define a random variable $R$ with a binomial distribution $(R \sim \text{Bin}(10,0.2))$.
    \begin{enumerate}
        \item Compute $\mathbb{E}(R)$
        \begin{enumerate}
            \item {\color{red} $10 \times 0.2 = 2$  }
        \end{enumerate} 
                
        \item Compute $V(R)$
        \begin{enumerate}
            \item {\color{red}  $10 \times 0.2 \times 0.8 = 1.6$  }
        \end{enumerate} 

        \item Describe to someone who may not have statistical expertise what $P(R=3)$ means? Be sure to include assumptions about the Binomial distribution and how the parameters $N,\theta$ relate to this probability.
        \begin{enumerate}
            \item {\color{red}  The probability that $R=3$ represents the probability of three successes among 10 trials where the probability of each trial equal 0.2.  }
        \end{enumerate} 

        \item For what value of $\theta$ is $V(R)$ the highest? Why does this make sense intuitively?
        \begin{enumerate}
            \item {\color{red} (Throwing Out This problem)  }
        \end{enumerate} 

    \end{enumerate}
    \item Suppose $Y \sim \text{Pois}(2)$
    \begin{enumerate}
        \item Compute $P(Y=2)$
        \begin{enumerate}
            \item {\color{red} $ \frac{e^{-2} 2^{2}}{2!} = 2e^{-2}$  }
        \end{enumerate} 

        \item Compute $P(Y \leq 2)$
        \begin{enumerate}
            \item {\color{red}
            \begin{align}
                P(Y=0) + P(Y=1) + P(Y=2) \\ 
                \frac{e^{-2} 2^{0}}{0!} + \frac{e^{-2} 2^{1}}{1!} + \frac{e^{-2} 2^{2}}{2!} \\
                e^{-2} + 2e^{-2} + 2e^{-2} = 5e^{-2}
            \end{align}
            }
        \end{enumerate} 

        \item Compute $P(Y > 2)$
        \begin{enumerate}
            \item {\color{red} 1- 5e^{-2}}
        \end{enumerate} 

    \end{enumerate}
    
    \item $X \sim \mathcal{N}(\mu, \sigma^{2}) $
    \begin{enumerate}
        \item Let $Y = X - \mu$. What is the distribution of $Y$?
        \begin{enumerate}
            \item {\color{red}  
            \begin{align}
                Y & \sim \mathcal{N}( \mu - \mu, \sigma^{2}) \\ 
                Y & \sim \mathcal{N}( 0, \sigma^{2}) \\ 
            \end{align}
            }
        \end{enumerate} 

        \item Let $Z = Y/\sigma$. What is the distribution of $Z$?
        \begin{enumerate}
            \item {\color{red}
            \begin{align}
                Z & \sim \mathcal{N}(0, \frac{1}{\sigma^{2}}\sigma^{2}) \\ 
                Z & \sim \mathcal{N}( 0, 1)
            \end{align}
                 
            }
        \end{enumerate} 

        \item Compute $P(Z = 0)$
        \begin{enumerate}
            \item {\color{red} 0  }
        \end{enumerate} 

    \end{enumerate}
    
    \item Suppose we define a new random variable $W$ with support $supp(W) = [0,1]$ and probability density function 
       \begin{align}
          f_{W}(w) = 2w
       \end{align}
      \begin{enumerate}
          \item Compute $P(W < 1/2) = \int_{0}^{1/2} f_{W}(w)\; dw$
        \begin{enumerate}
            \item {\color{red} Triangle with base 1/2 and height 1 so $\frac{1}{2} \time (\frac{1}{2} \times 1) = 1/4$  }
        \end{enumerate} 

          \item Compute $P(W < 1) = \int_{0}^{1} f_{W}(w)\; dw$
        \begin{enumerate}
            \item {\color{red}  1 }
        \end{enumerate} 

      \end{enumerate}
      
    \item Let $X$ be a continuous random variable. Is $P(X \leq x) = P(X < x)$? Why or why not? 
        \begin{enumerate}
            \item {\color{red}  Yes. 
            \begin{align}
                P(X \leq x) = P( [X < x] \cup [X=x] ) \\ 
                P(X<x) + P(X=x) \\ 
                P(X<x) + 0 \\ 
                P(X<x)
            \end{align}
         }
        \end{enumerate} 


\end{document}