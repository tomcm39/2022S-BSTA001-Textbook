\documentclass[krantz1,ChapterTOCs]{krantz}
\usepackage{fixltx2e,fix-cm}
\usepackage{amssymb}
\usepackage{amsmath}
\usepackage{graphicx}
\usepackage{subfigure}
\usepackage{makeidx}
\usepackage{multicol}
\usepackage{hyperref}
\usepackage{xcolor}

\begin{document}

\section{Exercises-Answer key}

\begin{enumerate}
    \item \begin{align*}
              A &= {1,2,3,4,5,6};
              \;\;B = {1,3,6} \\
              C &= {7}  
              \;\;D = \emptyset
           \end{align*}
    \begin{enumerate}
        \item Please compute $A \cap B$
        \begin{enumerate}
            \item {\color{red} $A \cap B = \{1,3,6  \}$ }
        \end{enumerate}
        \item Please compute $A \cup C$
            \begin{enumerate}
                \item {\color{red} $A \cup C =  \{1,2,3,4,5,6,7\}$}
            \end{enumerate}
        
        \item Please compute $A \cup D$
            \begin{enumerate}
                \item {\color{red} $A \cup D = \{1,2,3,4,5,6\}$  }
            \end{enumerate}
        
        \item Please compute $A \cap D$
            \begin{enumerate}
                \item {\color{red} $A \cap D = \emptyset$  }
            \end{enumerate}
        
        \item Please compute $(A \cap B) \cup (C \cup D)$
            \begin{enumerate}
                \item {\color{red}
                \begin{align*}
                    (A \cap B) \cup (C \cup D) &= \{1,3,6\} \cup \{7\} \\
                    &= \{1,3,6,7\}
                \end{align*}    
                    }
            \end{enumerate}
        
    \end{enumerate}
    \item Let the sample space $\mathcal{G} = \{ 1,2,3,4,5,6,7 \}$
    \begin{enumerate}
       \item Please compute $A^{c}$
           \begin{enumerate}
                \item {\color{red} $A^{c} = \{7\}$  }
            \end{enumerate}
        
       \item Please compute $B^{c}$
           \begin{enumerate}
                \item {\color{red} $B^{c} = \{2,4,5\}$  } 
            \end{enumerate}
        
       \item Please compute $D^{c}$
           \begin{enumerate}
                \item {\color{red} $D^{c} = \{1,2,3,4,5,6,7\}$  }
            \end{enumerate}
        
       \item Please compute $\mathcal{G} \cap A$
           \begin{enumerate}
                \item {\color{red} $ \mathcal{G} \cap A = \{ 1,2,3,4,5,6 \} $  }
            \end{enumerate}
        
       \item Is $A \subset \mathcal{G}$?
           \begin{enumerate}
                \item {\color{red} Yes, every item in A is a item in $\mathcal{G}$  }
            \end{enumerate}
        
       \item Is $\emptyset \subset \mathcal{G}$?
           \begin{enumerate}
                \item {\color{red} Yes, every item (nothing) in the empty set is in $\mathcal{G}$  }
            \end{enumerate}
        
    \end{enumerate}
    
    \item Let the sample space $\samplespace = \{0,1,2,a,b,c\}$ and let $A=\{0,1\}$, $B=\{x | x\text{ is a letter of the English alphabet}\}$
    \begin{enumerate}
        \item Please compute $A \cap B$
            \begin{enumerate}
                \item {\color{red} $ A \cap B =  \emptyset $  }
            \end{enumerate}
        
        \item Please compute $A \cup B$
            \begin{enumerate}
                \item {\color{red} $A \cup B = \{0,1,a,b,c\}$  }
            \end{enumerate}
        
        \item Please compute $A^{c}$
            \begin{enumerate}
                \item {\color{red} $A^{c} = \{2,a,b,c \} $  }
            \end{enumerate}
        
        \item Is $A \cup B = $\mathcal{G}$?
            \begin{enumerate}
                \item {\color{red} No, the item 2 is in $\mathcal{G}$ but this item is not in  $A \cap B$  }
            \end{enumerate}
        
        \item If we assigned probabilities to all outcomes, could $P(A \cup B) = 1$? why or why not?
            \begin{enumerate}
                \item {\color{red} No, only a set that contains all items (the sample space) can equal probability one. Because A union B is not equal to the sample space then it cannot be assigned a probability of one.  }
            \end{enumerate}
        
    \end{enumerate}
    
    \item Let $A = {0,1,2}$ for some sample space $\mathcal{G} = \{0,1,2,3,4,5,6\}$. Further assume $P(A) = 0.2$. 
    \begin{enumerate}
        \item Are the sets $A$ and $A^{c}$ disjoint? Why or why not.
            \begin{enumerate}
                \item {\color{red} Yes, by definition any item in $A$ cannot be an item in $A^{c}$ and so $A \cap A^{c} = \emptyset$}
            \end{enumerate}
        
        \item Simplify $P(A \cup A^{c})$ into an expression that involves $P(A)$ and $P(A^{c})$.
            \begin{enumerate}
                \item {\color{red} 
                \begin{align*}
                    P(A \cup A^{c}) = P(A) + P(A^{c}) & & \text{(Disjoint)}
                \end{align*}
                }
            \end{enumerate}
        
        \item Use Kolmogorov's axioms to show that $P(A) = 1 - P(A^{c})$ 
            \begin{enumerate}
                \item {\color{red}   
                \begin{align*}
                    P(\mathcal{G}) &= P(A \cup A^{c}) & & \text{(Every item is either in or not in A)} \\ 
                    P(\mathcal{G}) &= P(A) + P(A^{c}) & & \text{(Disjoint)} \\
                    1 &= P(A) + P(A^{c}) & & \text{(Probability of sample space)} \\ 
                   P(A)  &= 1 -  P(A^{c})
                \end{align*}
                }
            \end{enumerate}
        
    \end{enumerate}
    
    \item Let $\mathcal{G} = \{x | x\text{ is a positive integer}\}$
    \begin{enumerate}
        \item Are the sets $\emptyset$ and $\mathcal{G}$ disjoint?
            \begin{enumerate}
                \item {\color{red} Yes, the intersection of the empty set and the sample space is empty. }
            \end{enumerate}
        
        \item Simplify $P(\mathcal{G} \cup \emptyset)$ into an expression that involves $P(\mathcal{G})$ and $P(\emptyset)$
            \begin{enumerate}
                \item {\color{red} 
                \begin{align*}
                 P(\mathcal{G} \cup \emptyset) = P(\mathcal{G}) + P(\emptyset) & & \text{(Disjoint)}
                \end{align}
                }
            \end{enumerate}
        
        \item Use Kolmogorov's axioms to show that $P(\emptyset) = 0$
            \begin{enumerate}
                \item {\color{red}
                \begin{align*}
                1 &= P(\mathcal{G} \cup \emptyset) \\ 
                 P(\mathcal{G} \cup \emptyset) &= P(\mathcal{G}) + P(\emptyset) & & \text{(Disjoint)}\\
                1 &= P(\mathcal{G}) + P(\emptyset)\\
                P(\emptyset) &=  P(\mathcal{G}) - 1\\
                P(\emptyset) &=  1 - 1 = 0\\
                \end{align}
                }
            \end{enumerate}
        
    \end{enumerate}
    
    \item If $A=\{1,2,3\}$ and $B = \{2,3,4\}$ and $C = \{1,3\}$
    \begin{enumerate}
        \item Can $P(A) < P(B)$? Why or why not
            \begin{enumerate}
                \item {\color{red} The probability of $A$ could be smaller, but we cannot say for sure because $A$ is not a subset of $B$  }
            \end{enumerate}
        
        \item Can $P(A) < P(C)$? Why or why not
            \begin{enumerate}
                \item {\color{red} No, $C$ is a subset of $A$ and so the probability of $A$ cannot be smaller than the probability of $C$. }
            \end{enumerate}
        
    \end{enumerate}
    \item Use what you know about the intersection, about subsets, and about probability to show that $P(A \cap B) \leq P(A)$. Hint: How are $A \cap B$ and $A$ related?
        \begin{enumerate}
                \item {\color{red}
                \begin{align*}
                 A \cap B \subset A \\ 
                 P(A \cap B) \leq P(A) 
                 \end{align*}
                }
            \end{enumerate}
        
    \item Suppose we wish to study the reemergence of cancer among patients in remition. We collect data on 1,000 patients who are in cancer remition and follow them for 5 years. At five years we are interested in the probability of a second cancer. 
    \begin{enumerate}
        \item Define a sample space $\mathcal{G}$ we can use to assign probabilities to a second cancer and no second cancer.
            \begin{enumerate}
                \item {\color{red} $\mathcal{G} = \{ Yes, No\}$ }
            \end{enumerate}
        
        \item After five years of followup we find that 238 patients experienced a second cancer. Use the frequentist approach to assign probabilities to a second cancer \underline{and} no second cancer.
            \begin{enumerate}
                \item {\color{red}
                \begin{align*}
                    P(Yes) = 238/1000\\
                    P(No) = 762/1000\\
                \end{align*}
                
                }
            \end{enumerate}
        
        \item If you collected data on 2,000 patients do you expect the probability of a second cancer to change? How do you expect the probability to be different for 2,000 patients than with 1,000 patients? 
            \begin{enumerate}
                \item {\color{red} The probability will likely change and i expect that the probability assigned after collecting 2000 patients compared to 1000 patients would more accurately represent the true probability of a second cancer. }
            \end{enumerate}
        
    \end{enumerate}
    
   \item A study (link = \href{here}{https://www.science.org/doi/10.1126/science.abj8222} found that young adults were 32 times more at risk to develop multiple sclerosis (MS) after infection with the Epstein-Barr virus compared to young adults who were not infected by the virus. The experiment enrolled 10 million young adults and observed them for a period of 20 years.
   \begin{enumerate}
       \item Design a sample space if we wish to study outcomes that describe the number of young adults who develop MS. 
           \begin{enumerate}
                \item {\color{red} $\mathcal{G} = \{ 0,1,2,3,\cdots,10 \times 10^{6} \}$  }
            \end{enumerate}
        
       \item Build the event $(E_{1})$ "less than 10\% of young adults develop MS" using set notation.
           \begin{enumerate}
                \item {\color{red} $E_{1} = \{ x | x \leq 10^{6}  \}$ }
            \end{enumerate}
        
       \item Build the event $(E_{2})$ "less than 5\% of young adults develop MS" using set notation.
           \begin{enumerate}
                \item {\color{red} $E_{2} = \{ x | x \leq 5\times 10^{5}  \}$}
            \end{enumerate}
        
       \item Are $E_{1}$ and $E_{2}$ disjoint? Why or why not?
           \begin{enumerate}
                \item {\color{red} No, because $E_{1}$ contains items in $E_{2}$}
            \end{enumerate}
        
       \item Can $P(E_{1}) < P(E_{2})$?
           \begin{enumerate}
                \item {\color{red} No, because $E_{2}$ is a subset of $E_{1}$ }
            \end{enumerate}
        
   \end{enumerate}
    
\end{document}
